
%Обзор, введение в тему, обозначение места данной работы в
%мировых исследованиях и~т.\:п., можно использовать ссылки на~другие
%работы~\autocite{Gosele1999161,Lermontov}
%(если их~нет, то~в~автореферате
%автоматически пропадёт раздел <<Список литературы>>). Внимание! Ссылки
%на~другие работы в~разделе общей характеристики работы можно
%использовать только при использовании \verb!biblatex! (из-за технических
%ограничений \verb!bibtex8!. Это связано с тем, что одна
%и~та~же~характеристика используются и~в~тексте диссертации, и в
%автореферате. В~последнем, согласно ГОСТ, должен присутствовать список
%работ автора по~теме диссертации, а~\verb!bibtex8! не~умеет выводить в~одном
%файле два списка литературы).
%При использовании \verb!biblatex! возможно использование исключительно
%в~автореферате подстрочных ссылок
%для других работ командой \verb!\autocite!, а~также цитирование
%собственных работ командой \verb!\cite!. Для этого в~файле
%\verb!common/setup.tex! необходимо присвоить положительное значение
%счётчику \verb!\setcounter{usefootcite}{1}!.
%Настроить размер изображения с логотипом можно
%в~соответствующих местах файлов \verb!title.tex!  отдельно для
%диссертации и автореферата. Если вам логотип не~нужен, то просто
%удалите файл с~логотипом.

%\ifsynopsis
%Этот абзац появляется только в~автореферате.
%Для формирования блоков, которые будут обрабатываться только в~автореферате,
%заведена проверка условия \verb!\!\verb!ifsynopsis!.
%Значение условия задаётся в~основном файле документа (\verb!synopsis.tex! для
%автореферата).
%\else
%Этот абзац появляется только в~диссертации.
%Через проверку условия \verb!\!\verb!ifsynopsis!, задаваемого в~основном файле
%документа (\verb!dissertation.tex! для диссертации), можно сделать новую
%команду, обеспечивающую появление цитаты в~диссертации, но~не~в~автореферате.
%\fi

%В папке Documents можно ознакомиться в решением совета из Томского ГУ
%в~файле \verb+Def_positions.pdf+, где обоснованно даются рекомендации
%по~формулировкам защищаемых положений.


{\actuality} Системы идентификации автомобилей и регистрации нарушений правил дорожного движения играют важную роль в управлении безопасностью на дорогах, способствуют повышению неотвратимости наказания и снижению смертности в авариях. Существующие системы основаны на видео- или фото-идентификации номеров автомобилей. Работе этих систем могут препятствовать загрязнение при плохих погодных условиях или специальное заслонение номерного знака. Из-за этого, по сведениям ГИБДД, производительность систем идентификации автомобилей может падать ниже 50~\%. Для борьбы с этим недостатком можно использовать метод радиочастотной идентификации (RFID). При использовании RFID в номерной знак или под лобовое стекло автомобиля помещается специальная радиометка, которая, находясь в области действия RFID-считывателя, передает ему свой идентификатор. Особый интерес представляют пассивные системы RFID УВЧ-диапазона 860--960~МГц, в которых метки не обладают собственными источниками энергии и могут быть идентифицированы считывателями на расстоянии порядка 10--20 метров.

Технология RFID давно используется в различных приложениях, связанных с транспортом, однако в этой области по-прежнему остается много неисследованных вопросов. Производительность RFID существенно зависит как от особенностей окружения, распространения радиосигналов и антенн, так и от параметров протокола связи между считывателем и метками международного стандарта EPC Class 1 Generation 2 (EPC Gen2). Хотя анализу производительности систем радиочастотной идентификации в различных приложениях посвящено немало работ, тема применения RFID для идентификации автомобилей остается менее изученной с теоретической точки зрения. Для получения адекватной оценки производительности необходимо создание моделей, учитывающих как параметры распространения радиосигналов, так и всевозможные настройки считывателей и параметры протокола.

Для работы автоматической системы безопасности необходимо соединять точки идентификации с центрами обработки данных компьютерными сетями, по которым оперативно передается информация о распознанных автомобилях. Проводные решения оказываются не всегда доступными по техническим или экономическим причинам, поэтому особый интерес представляют беспроводные сети. Такие сети имеют низкую стоимость и быстро строятся, однако их производительность может быть невысокой. Для анализа проивзодительности беспроводных сетей часто применяются стохастические методы, но из-за высокой сложности их зачастую трудно применять для анализа больших многошаговых сетей. Одним из наиболее распространённых методов исследований являются сети массового обслуживания (СеМО), в которых каналы связи и маршрутизаторы моделируются обслуживающими приборами с очередями. Хорошую точность при описании реальных систем показывают СеМО с марковсиким входящими потоками (MAP) и распределением времени обслуживания фазового типа (PH). Хотя системы MAP/PH/1/N хорошо исследованы, их применение для построения открытых сетей массового обслуживания, моделирующих реальные беспроводные сети, изучено в меньшей степени. Из-за высокой сложности моделей особый интерес представляют методы быстрого получения приближенных оценок их характеристик.

Для реализации системы радиочастотной идентификации транспорта нужны системы управления и программное обеспечение для связи считывателей с центрами обработки данных. Эти системы обладают рядом особенностей. Во-первых, считыватели могут исчисляться сотнями или тысячами, а передача данных от них должна идти непрерывно. Во-вторых, нужно объединять данные о прочитанных метках с данными от камер. Наконец, для различных служб и ведомств система должна давать разные уровени доступа. Существует множество различных систем и подходов к построению промежуточного программного обеспечения и систем управления считывателями. Кроме исследовательских работ существует ряд стандартов, а также множество готовых коммерческих решений. Однако, идентификации транспорта обладает рядом специфических особенностей, делающих актуальной задачей разработку распределенной системы управления считывателями.

Исследованиям в области разработки и исследования широкополосных беспроводных сетей и технологии RFID, методам теории массового обслуживания и вопросам ее применения и прочим темам, затрагиваемым в диссертации, посвящен ряд работ, среди которых следует особо отметить работы отечественных и зарубежных учёных: В.М. Вишневский, А.Н. Дудин, П.В. Никитин, А.Е. Кучерявый, К.Е. Самуйлов, Ю.В. Гайдамака, В.В. Рыков, Р.Н. Минниханов, D. Lucantoni, N. Abramson, L.G. Roberts, M.F. Neuts, G. Horvath, G. Bianchi,  H. Okamura, P. Buchholz, M. Telek, J.C. Strelen, L. Bodrog, D. Aldous, K.V.S. Rao, C. Floerkemeier, C. Wang, E. Vahedi, R. Nelson, T.S. Rappaport, S. Singh, J. Kriege, P. Djukic, S. Valaee, J.E. Hoag. и др.

% {\progress}
% Этот раздел должен быть отдельным структурным элементом по
% ГОСТ, но он, как правило, включается в описание актуальности
% темы. Нужен он отдельным структурынм элемементом или нет ---
% смотрите другие диссертации вашего совета, скорее всего не нужен.

{\object} являются системы радиочастотной идентификации транспорта и опорные многошаговые беспроводные сети. {\objective} являются методы анализа и алгоритмы расчета характеристик производительности систем радиочастотной идентификации и многошаговых беспроводных сетей, а также методы построения систем управления распределенными системами идентификации транспорта.

{\aim} данной работы является создание комплекса моделей для анализа производительности систем радиочастотной идентификации и широкополосных беспроводных сетей, а также разработка и экспериментальное внедрение системы управления и сбора данных с RFID-считывателей.

Для~достижения поставленной цели было необходимо решить следующие {\tasks}:
\begin{enumerate}[beginpenalty=10000] % https://tex.stackexchange.com/a/476052/104425
  \item Разработать комплекс аналитических и имитационных моделей для анализа производительности систем радиочастотной идентификации транспортных средств, позволяющих адекватно оценивать их производительность.
  \item Разработать комплекс аналитических моделей на базе методов теории массового обслуживания для оценки производительности опорных беспроводных сетей с использованием марковских случайных потоков для моделирования трафика и распределений фазового типа для моделирования процесса передачи данных, а также методов понижения размерности для нахождения эффективных способов расчёта показателей производительности.
  \item Разработать распределенную систему управления и сбора данных с RFID-считывателей и провести её экспериментальное внедрение.
\end{enumerate}


{\novelty}
\begin{enumerate}[beginpenalty=10000] % https://tex.stackexchange.com/a/476052/104425
  \item Впервые предложена стохастическая модель системы радиочастотной идентификации с мобильными метками, учитывающая различные законы движения меток, а также сценарии переключения питания и смены опрашиваемых значений флагов сессий.
  \item Предложен комплекс новых аналитических и имитационных моделей для анализа вероятности идентификации быстро движущихся транспортных средств с учетом особенностей логического и физического уровней EPC Gen2, а также учитывающих особенности распространения радиосигналов вблизи автодороги.
  \item Предложена методика построения сетей массового обслуживания с марковскими входными потоками и распределениями фазового типа для адекватного моделирования многошаговых беспроводных сетей, учитывающая различия в трафике и интерференции, испытываемые разными станциями сети.
  \item Предложен и исследован метод приближенного вычисления оценок характеристик тандемных открытых сетей массового обслуживания с коррелированными марковскими входными потоками и распределениями обслуживания фазового типа, основанный на аппроксимации потоков обслуженных заявок PH-распределениями или MAP-потоками по первым моментам и автокорреляции. В численном эксперименте показано, что предложенный метод позволяет находить оценки характеристик сети с высокой точностью за малое время.
  \item Разработана и реализована распределенная система управления RFID-считывателями, предназначенная для организации сбора данных об идентифицированных транспортных средствах и допускающая объединение с данными, поступающими от прочих источников идентификации.
  \item Проведено три эксперимента по внедрению системы радиочастотной идентификации на дорогах в г. Казань и на ЦКАД в Московской области, получены экспериментальные данные об эффективности разработанной системы.
\end{enumerate}

{\speciality}
Диссертационная работа соответствует содержанию специальности 05.13.15, а именно исследованиям и разработке научных основ архитектурных, структурных, логических и технических принципов создания вычислительных машин, комплексов и компьютерных сетей, организации арифметической, логической, символьной и специальной обработки данных, хранения и ввода-вывода информации, параллельной и распределенной обработки информации, многопроцессорных и многомашинных вычислительных систем, сетевых протоколов и служб передачи данных в компьютерных сетях, взаимодействия и защиты компьютерных сетей. В выполненной работе
присутствуют оригинальные результаты одновременно из четырех областей: разработка принципов создания компьютерных сетей, математического моделирования копмьютерных сетей, численных методов и комплексов программ. Диссертационная работа соответствует следующим пунктам специальности:
\begin{itemize}
    \item Разработка научных основ создания вычислительных машин, комплексов и компьютерных сетей, исследования общих свойств и принципов функционирования вычислительных машин, комплексов и компьютерных сетей.
    \item Теоретический анализ и экспериментальное исследование функционирования вычислительных машин, комплексов и компьютерных сетей с целью улучшения их технико-экономических и эксплуатационных характеристик.
    \item Разработка научных методов и алгоритмов организации параллельной и распределенной обработки информации, многопроцессорных, многомашинных и специальных вычислительных систем.
    \item Разработка научных методов и алгоритмов создания структур и топологий компьютерных сетей, сетевых протоколов и служб передачи данных в компьютерных сетях, взаимодействия компьютерных сетей, построенных с использованием различных телекоммуникационных технологий, мобильных и специальных компьютерных сетей, защиты компьютерных сетей и приложений.
\end{itemize}

{\influence}
Аналитические модели протокола EPC Gen2, предложенные в работе, могут использоваться для первоначальной оценки производительности системы радиочастотной идентификации автомобилей. Имитационная модель этой системы, учитывающая особенности протокола, оборудования и канала связи между считывателем и метками, может использоваться для получения оценок производительности системы при различных интенсивностях и скоростях дорожного движения.

Модели многошаговых беспроводных сетей могут использоваться для оценки производительности (межконцевых задержек и вероятностей потери пакетов) при проектировании телекоммуникационных сетей.

Распределенная система управления считывателями и промежуточное программное обеспечение, описанные в работе, использовались в трех экспериментальных внедрениях. Предложенные решения могут использоваться при дальнейшем практическом внедрении системы радиочастотной идентификации автомобилей.

Результаты работы также были использованы в исследованиях, проводимых по следующим грантам:

\begin{itemize}
    \item Контракт c Министерством образования и науки РФ № 14.514.11.4071 в рамках федеральной целевой программы <<Исследования и разработки по приоритетным направлениям развития научно-технологического комплекса России на 2007-2013 годы>>.
	\item Соглашение с Министерством образования и науки РФ о предоставлении субсидии от 22.10.2014 г. № 14.613.21.0020 в рамках федеральной целевой программы <<Исследования и разработки по приоритетным направлениям развития научно-технологического комплекса на 2014-2020 годы>>.
	\item Грант Российского научного фонда (РНФ) № 16-49-02021.
	\item Грант Российского фонда фундаментальных исследований (РФФИ) № 13-07-00737.
	\item Грант Российского фонда фундаментальных исследований (международный проект РФФИ "--- БРФФИ) № 14-07-90015.
	\item Грант Российского фонда фундаментальных исследований (международный проект РФФИ "--- БРФФИ) № 16-57-00130.

	% \item Контракт c Министерством образования и науки РФ № 14.514.11.4071 в рамках федеральной целевой программы <<Исследования и разработки по приоритетным направлениям развития научно-технологического комплекса России на 2007-2013 годы>> поисковые научно-исследовательские работы по лоту шифр <<2013-1.4-14-514-0122>> <<Разработка технологии и архитектуры аппаратно-программных средств сверхвысокоскоростных беспроводных сетей в миллиметровом диапазоне радиоволн 60-100 ГГц>> по теме: <<Разработка и исследование технологии и архитектуры сверхвысокоскоростных беспроводных сетей в миллиметровом Е-диапазоне радиоволн 71-76 ГГц, 81-86 ГГц>>
	% \item Соглашение с Министерством образования и науки РФ о предоставлении субсидии от 22.10.2014 г. № 14.613.21.0020 в рамках федеральной целевой программы <<Исследования и разработки по приоритетным направлениям развития научно-технологического комплекса на 2014-2020 годы>> по теме: <<Разработка математических методов, алгоритмов и программ оценки производительности и проектирования широкополосных беспроводных сетей передачи мультимедийной информации вдоль протяженных транспортных магистралей (железнодорожное полотно, автодороги) и трубопроводов (нефтяные и газовые магистрали)>>
	% \item Грант Российского научного фонда (РНФ) № 16-49-02021 <<Новый комплекс математических моделей, методов, алгоритмов и программ управляемых стохастических систем для оценки производительности и проектирования телекоммуникационных сетей следующего поколения>>
	% \item Грант Российского фонда фундаментальных исследований (РФФИ) № 13-07-00737 <<Разработка принципов построения и математических методов исследования нового класса автоматизированных систем безопасности на автодорогах с использованием RFID-технологий и высокоскоростной беспроводной связи>>
	% \item Грант Российского фонда фундаментальных исследований (международный проект РФФИ "--- БРФФИ) № 14-07-90015 Бел\_а <<Разработка и исследование методов оценки производительности и проектирования гибридных систем передачи мультимедийной информации на базе лазерной и радио технологий>>
	% \item Грант Российского фонда фундаментальных исследований (международный проект РФФИ "--- БРФФИ) № 16-57-00130 Бел\_а <<Разработка и исследование методов синтеза архитектуры широкополосных беспроводных сетей с линейной топологией>>
\end{itemize}

{\methods} Для решения задач, поставленных в диссертации, использовались методы теории вероятности, математической статистики, теории случайных процессов, теории массового обслуживания, методы моделирования беспроводных протоколов и сетей с помощью марковских цепей, методы дискретно-событийного имитационного моделирования. При разработке программного обеспечения использовались методоы многопоточного программирования, методы разработки распределенных систем.

{\defpositions}
\begin{enumerate}[beginpenalty=10000] % https://tex.stackexchange.com/a/476052/104425
  \item Стохастическая модель системы радиочастотной идентификации с мобильными метками, учитывающая различные законы движения меток, а также сценарии переключения питания и смены опрашиваемых значений флагов сессий.
  \item Комплекс аналитических и имитационных моделей для анализа вероятности идентификации быстро движущихся транспортных средств с учетом особенностей логического и физического уровней EPC Gen2, а также особенностей распространения радиосигналов вблизи автодороги.
  \item Метод построения моделей открытых тандемных сетей массового обслуживания для анализа производительности многошаговых беспроводных сетей.
  \item Метод расчета оценок характеристик открытых сетей массового обслуживания с марковскими входящими потоками и обслуживанием фазового типа с использованием аппроксимации потоков обслуженных пакетов PH-распределениями и MAP-потоками.
  \item Реализация распределенной системы управления RFID-считывателями, используемой для создания систем радиочастотной идентификации автотранспорта.
\end{enumerate}

{\reliability} полученных результатов обеспечивается использованием строгих математических моделей, сравнением результатов аналитического и имитационного моделирования. Результаты анализа вероятности идентификации автомобилей согласуются с результатами, полученными в ходе реальных экспериментов. Результаты находятся в соответствии с результатами, полученными другими авторами.


{\probation}
Основные результаты работы докладывались~на следующих конференциях: 12th Annual IEEE International conference on RFID 2018 (IEEE RFID 2018; США, Орландо); 11th Annual IEEE International conference on RFID 2017 (IEEE RFID 2017; США, Финикс); международный форум Kazan Digital Week 2020 (Казань); International conference on Advances in Applied Probability and Stochastic Processes 2019 (ICAAP \& SP 2019; Индия, Коттаям); 13-е и 12-е Всероссийские совещания по проблемам управления (ВСПУ 2019, ВСПУ 2014; Москва, ИПУ РАН); 20-я, 18-я, 16-я и 15-я Международные конференции им. А.Ф. Терпугова Информационные технологии и математическое моделирование (ИТММ 2021, ИТММ 2019, ИТММ 2017, ИТММ 2016; Россия); 11th IEEE International Conference on Application of Information and Communication Technologies (IEEE AICT 2017; Москва, ИПУ РАН); 21-я, 20-я, 19-я, 18-я и 17-я Международные Конференции <<Распределенные компьютерные и телекоммуникационные сети: управление, вычисление, связь>> (DCCN 2018, DCCN 2017, DCCN 2016, DCCN 2015, DCCN 2013; Москва); 7th International Workshop on Communication Technologies for Vehicles (Nets4Cars-Fall 2014; Санкт-Петербург); 2012 International Conference on RFID-Technology and Applications (IEEE RFID-TA 2012; Франция, Ницца); 10-я и 5-я Всероссийская конференция <<Информационно-телекоммуникационные технологии и математическое моделирование высокотехнологичных систем>> (ИТТММ 2020, ИТТММ 2015; Москва, РУДН).

{\contribution} Автор принимал активное участие в разработке стохастической модели протокола EPC Gen2, имитационной модели системы радиочастотной идентификации автомобилей, аналитической и имитационной модели для расчёта производительности многошаговых беспроводных сетей, методов расчета характеристик тандемных сетей массового обслуживания, промежуточного программного обеспечения и системы управления считывателями, разработке считывателя и проведении экспериментов.

\ifnumequal{\value{bibliosel}}{0}
{%%% Встроенная реализация с загрузкой файла через движок bibtex8. (При желании, внутри можно использовать обычные ссылки, наподобие `\cite{vakbib1,vakbib2}`).
    {\publications} Основные результаты по теме диссертации изложены
    в~XX~печатных изданиях,
    X из которых изданы в журналах, рекомендованных ВАК,
    X "--- в тезисах докладов.
}%
{%%% Реализация пакетом biblatex через движок biber
    \begin{refsection}[bl-author, bl-registered]
        % Это refsection=1.
        % Процитированные здесь работы:
        %  * подсчитываются, для автоматического составления фразы "Основные результаты ..."
        %  * попадают в авторскую библиографию, при usefootcite==0 и стиле `\insertbiblioauthor` или `\insertbiblioauthorgrouped`
        %  * нумеруются там в зависимости от порядка команд `\printbibliography` в этом разделе.
        %  * при использовании `\insertbiblioauthorgrouped`, порядок команд `\printbibliography` в нём должен быть тем же (см. biblio/biblatex.tex)
        %
        % Невидимый библиографический список для подсчёта количества публикаций:
        \printbibliography[heading=nobibheading, section=1, env=countauthorvak,          keyword=biblioauthorvak]%
        \printbibliography[heading=nobibheading, section=1, env=countauthorwos,          keyword=biblioauthorwos]%
        \printbibliography[heading=nobibheading, section=1, env=countauthorscopus,       keyword=biblioauthorscopus]%
        \printbibliography[heading=nobibheading, section=1, env=countauthorconf,         keyword=biblioauthorconf]%
        \printbibliography[heading=nobibheading, section=1, env=countauthorother,        keyword=biblioauthorother]%
        \printbibliography[heading=nobibheading, section=1, env=countregistered,         keyword=biblioregistered]%
        \printbibliography[heading=nobibheading, section=1, env=countauthorpatent,       keyword=biblioauthorpatent]%
        \printbibliography[heading=nobibheading, section=1, env=countauthorprogram,      keyword=biblioauthorprogram]%
        \printbibliography[heading=nobibheading, section=1, env=countauthor,             keyword=biblioauthor]%
        \printbibliography[heading=nobibheading, section=1, env=countauthorvakscopuswos, filter=vakscopuswos]%
        \printbibliography[heading=nobibheading, section=1, env=countauthorscopuswos,    filter=scopuswos]%
        %
        \nocite{*}%
        %
        {\publications} Основные результаты по теме диссертации изложены в~\arabic{citeauthor}~печатных изданиях,
        \arabic{citeauthorvak} из которых изданы в журналах, рекомендованных ВАК\sloppy%
        \ifnum \value{citeauthorscopuswos}>0%
            , \arabic{citeauthorscopuswos} "--- в~изданиях, индексируемых Web of~Science и Scopus\sloppy%
        \fi%
        \ifnum \value{citeauthorconf}>0%
            , \arabic{citeauthorconf} "--- в~тезисах докладов.
        \else%
            .
        \fi%
        \ifnum \value{citeregistered}=1%
            \ifnum \value{citeauthorpatent}=1%
                Зарегистрирован \arabic{citeauthorpatent} патент.
            \fi%
            \ifnum \value{citeauthorprogram}=1%
                Зарегистрирована \arabic{citeauthorprogram} программа для ЭВМ.
            \fi%
        \fi%
        \ifnum \value{citeregistered}>1%
            Зарегистрированы\ %
            \ifnum \value{citeauthorpatent}>0%
            \formbytotal{citeauthorpatent}{патент}{}{а}{}\sloppy%
            \ifnum \value{citeauthorprogram}=0 . \else \ и~\fi%
            \fi%
            \ifnum \value{citeauthorprogram}>0%
            \formbytotal{citeauthorprogram}{программ}{а}{ы}{} для ЭВМ.
            \fi%
        \fi%
        % К публикациям, в которых излагаются основные научные результаты диссертации на соискание учёной
        % степени, в рецензируемых изданиях приравниваются патенты на изобретения, патенты (свидетельства) на
        % полезную модель, патенты на промышленный образец, патенты на селекционные достижения, свидетельства
        % на программу для электронных вычислительных машин, базу данных, топологию интегральных микросхем,
        % зарегистрированные в установленном порядке.(в ред. Постановления Правительства РФ от 21.04.2016 N 335)
    \end{refsection}%
    \begin{refsection}[bl-author, bl-registered]
        % Это refsection=2.
        % Процитированные здесь работы:
        %  * попадают в авторскую библиографию, при usefootcite==0 и стиле `\insertbiblioauthorimportant`.
        %  * ни на что не влияют в противном случае
        \nocite{RFID_JRFID2017}
        \nocite{WINET_IJPAM2016}
        \nocite{WINET_TCOMM2015}
        \nocite{QS_JPU2013}
        \nocite{QS_JITCS2013}
        \nocite{QS_TCOMM2012}

        \nocite{QS_ICAAPSP2020}
        \nocite{QS_ITMM2019}
        \nocite{RFID_IEEERFID2018}
        \nocite{RFID_SYNCHROINFO2018}
        \nocite{RFID_IEEERFID2017}
        \nocite{QS_AICT2017}
        \nocite{QS_ITMM2017}
        \nocite{QS_ITMM2016}
        \nocite{QS_DCCN2016_CCIS}
        \nocite{RFID_DCCN2016_CCIS}
        \nocite{RFIDCTRL_NETS2CARS2014}
        \nocite{RFIDTA2012}

        % \nocite{vakbib2}%vak
        % \nocite{patbib1}%patent
        % \nocite{progbib1}%program
        % \nocite{bib1}%other
        % \nocite{confbib1}%conf
    \end{refsection}%
        %
        % Всё, что вне этих двух refsection, это refsection=0,
        %  * для диссертации - это нормальные ссылки, попадающие в обычную библиографию
        %  * для автореферата:
        %     * при usefootcite==0, ссылка корректно сработает только для источника из `external.bib`. Для своих работ --- напечатает "[0]" (и даже Warning не вылезет).
        %     * при usefootcite==1, ссылка сработает нормально. В авторской библиографии будут только процитированные в refsection=0 работы.
}

% При использовании пакета \verb!biblatex! будут подсчитаны все работы, добавленные
% в файл \verb!biblio/author.bib!. Для правильного подсчёта работ в~различных
% системах цитирования требуется использовать поля:
% \begin{itemize}
%         \item \texttt{authorvak} если публикация индексирована ВАК,
%         \item \texttt{authorscopus} если публикация индексирована Scopus,
%         \item \texttt{authorwos} если публикация индексирована Web of Science,
%         \item \texttt{authorconf} для докладов конференций,
%         \item \texttt{authorpatent} для патентов,
%         \item \texttt{authorprogram} для зарегистрированных программ для ЭВМ,
%         \item \texttt{authorother} для других публикаций.
% \end{itemize}
% Для подсчёта используются счётчики:
% \begin{itemize}
%         \item \texttt{citeauthorvak} для работ, индексируемых ВАК,
%         \item \texttt{citeauthorscopus} для работ, индексируемых Scopus,
%         \item \texttt{citeauthorwos} для работ, индексируемых Web of Science,
%         \item \texttt{citeauthorvakscopuswos} для работ, индексируемых одной из трёх баз,
%         \item \texttt{citeauthorscopuswos} для работ, индексируемых Scopus или Web of~Science,
%         \item \texttt{citeauthorconf} для докладов на конференциях,
%         \item \texttt{citeauthorother} для остальных работ,
%         \item \texttt{citeauthorpatent} для патентов,
%         \item \texttt{citeauthorprogram} для зарегистрированных программ для ЭВМ,
%         \item \texttt{citeauthor} для суммарного количества работ.
% \end{itemize}
% % Счётчик \texttt{citeexternal} используется для подсчёта процитированных публикаций;
% % \texttt{citeregistered} "--- для подсчёта суммарного количества патентов и программ для ЭВМ.

% Для добавления в список публикаций автора работ, которые не были процитированы в
% автореферате, требуется их~перечислить с использованием команды \verb!\nocite! в
% \verb!Synopsis/content.tex!.