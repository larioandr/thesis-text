%% Согласно ГОСТ Р 7.0.11-2011:
%% 5.3.3 В заключении диссертации излагают итоги выполненного исследования, рекомендации, перспективы дальнейшей разработки темы.
%% 9.2.3 В заключении автореферата диссертации излагают итоги данного исследования, рекомендации и перспективы дальнейшей разработки темы.
\begin{enumerate}
  \item Описан способ построения распределенной системы радиочастотной идентификации автомобилей, выделены факторы, влияющие на производительность системы;
  \item Представлены результаты численного исследования аналитической модели протокола EPC Gen2, показывающие целесообразность периодической смены считывателем опрашиваемого значения флага сессии и периодического сброса питания;
  \item Предложена имитационная модель системы радиочастотной идентификации автомобилей, учитывающая особенности распространения сигнала вблизи дороги, эффект Допплера, параметры антенных систем, настройки протокола EPC Gen.2;
  \item С помощью имитационного моделирования получены численные оценки вероятности идентификации быстро движущихся автомобилей при различных настройках протокола. Полученные результаты близко соотносятся с данными, полученными в ходе экспериментального внедрения системы, и показывают, что при определенных настройках можно добиться высокой вероятности идентификации быстро движущихся автомобилей с размещенными в номерных знаках метках;
  \item Описан способ построения распределений фазового типа для моделирования беспроводных каналов связи с использованием существующих стохастических моделей каналов;
  \item Предложена открытая сеть массового обслуживания с марковскими входными потоками, обслуживанием фазового типа и ограниченной памятью, повторными передачами и потерями пакетов из-за ошибок в канале, позволяющая моделировать беспроводную опорную сеть;
  \item Для беспроводных сетей с линейной топологией предложен метод оценки производительности с использованием систем массового обслуживания MAP/PH/1/n;
  \item Рассмотрен способ расчёта параметров производительности тандемных сетей массового обслуживания с узлами вида MAP/PH/1/n, позволяющий ограничить размерность модели за счет использования методов аппроксимации MAP-потоков обслуженных пакетов;
  \item Представлены численные результаты, показывающие адекватную точность предложенных методов моделирования беспроводных каналов связи и опорных беспроводных сетей с линейной топологией;
  \item Предложена архитектура промежуточного программного обеспечения для подключения большого количества географически разделенных RFID-считывателей к центру управления, обеспечивающего получение данных от считывателей, мониторинг и управление системой;
  \item Представлены результаты экспериментов по исследованию и внедрению системы радиочастотной иденификации автобусов в городе Казань 2014 и 2020 годов. Показано, что эффективность системы достигает 95~\%.
\end{enumerate}
