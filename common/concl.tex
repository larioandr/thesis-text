%% Согласно ГОСТ Р 7.0.11-2011:
%% 5.3.3 В заключении диссертации излагают итоги выполненного исследования, рекомендации, перспективы дальнейшей разработки темы.
%% 9.2.3 В заключении автореферата диссертации излагают итоги данного исследования, рекомендации и перспективы дальнейшей разработки темы.
\begin{enumerate}
  \item Описан способ построения распределенной системы радиочастотной идентификации автомобилей, выделены факторы, влияющие на производительность системы.
  \item Представлены результаты численного исследования аналитической модели UHF RFID, показывающие целесообразность периодической смены считывателем опрашиваемого значения флага сессии и периодического сброса питания.
  \item Предложена имитационная модель системы радиочастотной идентификации автомобилей, учитывающая особенности распространения сигнала вблизи дороги, эффект Доплера, параметры антенных систем, настройки протокола EPC Class 1 Gen.2.
  \item С помощью имитационного моделирования получены численные оценки вероятности идентификации быстро движущихся автомобилей при различных настройках протокола. Полученные результаты показывают, что возможно добиться высокой вероятности идентификации при размещении меток в номерных знаках.
  \item Предложена методика построения открытых сетей массового обслуживания с узлами MAP/PH/1/N, адекватно моделирующих многошаговые беспроводные сети с каналами IEEE 802.11. Для обеспечения высокой точности модели, при поиске распределений времени обслуживания используются данные имитационного моделирования беспроводных каналов связи. Предложенная методика обеспечивает точность свыше 80~\% при многократном увеличении скорости расчета.
  \item Предложен метод приближенного расчёта характеристик тандемных сетей массового обслуживания с узлами MAP/PH/1/N,   позволяющий ограничить пространство состояний модели за счет использования методов аппроксимации выходящих MAP-потоков. Аппроксимация производится PH-распределениями по 1, 2 или 3 моментам, а также MAP-потоками по трем моментам и коэффициенту корреляции. Аппроксимация PH-распределениями по трем моментам позволяет с высокой точностью получить оценки характеристик сетей с большим числом узлов, требуя меньше времени, чем метод Монте-Карло.
  \item Предложена архитектура распределенной системы управления и промежуточного программного обеспечения для подключения RFID-считывателей к центру управления, обеспечивающего получение данных от считывателей, мониторинг и управление системой.
  \item Представлены результаты экспериментов по исследованию и внедрению системы радиочастотной иденификации автобусов в городе Казань 2014 и 2020 годов, а также в Московской области в 2021 году. Показано, что эффективность системы достигает 95~\%.
\end{enumerate}
