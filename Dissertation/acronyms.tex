\chapter*{Список сокращений и условных обозначений} % Заголовок
\addcontentsline{toc}{chapter}{Список сокращений и условных обозначений}  % Добавляем его в оглавление
% при наличии уравнений в левой колонке значение параметра leftmargin приходится подбирать вручную
\begin{description}[align=right,leftmargin=3.5cm]
% \item[%
%     \(\begin{rcases}
%         a_n\\
%         b_n
%     \end{rcases}\)%
%     ] коэффициенты разложения Ми в дальнем поле соответствующие
% электрическим и магнитным мультиполям
% \item[%
%     \({\boldsymbol{\hat{\mathrm e}}}\)%
%     ] единичный вектор
% \item[\(E_0\)] амплитуда падающего поля
% \item[\(j\)] тип функции Бесселя
% \item[\(k\)] волновой вектор падающей волны
% \item[%
%     \(\begin{rcases}
%         a_n\\
%         b_n
%     \end{rcases}\)%
%     ] и снова коэффициенты разложения Ми в дальнем поле соответствующие
% электрическим и магнитным мультиполям. Добавлено много текста, так что описание группы условных
% обозначений значительно превысило высоту этой группы...
% \item[\(L\)] общее число слоёв
% \item[\(l\)] номер слоя внутри стратифицированной сферы
% \item[\(\lambda\)] длина волны электромагнитного излучения
% в вакууме
% \item[\(n\)] порядок мультиполя
% \item[%
%     \(\begin{rcases}
%         {\mathbf{N}}_{e1n}^{(j)}&{\mathbf{N}}_{o1n}^{(j)}\\
%         {\mathbf{M}_{o1n}^{(j)}}&{\mathbf{M}_{e1n}^{(j)}}
%     \end{rcases}\)%
%     ] сферические векторные гармоники
% \item[\(\mu\)] магнитная проницаемость в вакууме
% \item[\(r,\theta,\phi\)] полярные координаты
% \item[\(\omega\)] частота падающей волны
\item[ALE] Application Level Events, события уровня приложений
\item[API] Application Programming Interface, программный интерфейс приложения
\item[ASK] Amplitude Shift Keying, амплитудная манипуляция
\item[AWGN] Additive white Gaussian noise, аддитивный белый гауссовский шум
\item[BER] Bit Error Rate, вероятность битовой ошибки
\item[BLF] Backscatter Link Frequency, частота обратного рассеяния
\item[CRC] Cyclic Redundancy Check, циклический избыточный код
\item[CSMA/CA] Collision Sense Multiple Access with Collision Avoidance, множественный доступ с контролем несущей и избежанием коллизий
\item[CTS] Clear to Send, подтверждение доступности канала для передачи
\item[CW] в контексте RFID: Constant Wave, постоянный синусоидальный сигнал; в контексте CSMA/CA: Contention Window, размер конкурентного окна
\item[DCF] Distributed Coordination Function, распределенная функция координации
\item[DCI] Discovery, Configuration, and Initialization, поиск, настройка и инициализация
\item[DIFS] DCF Interframe Space, межкадровый интервал в DCF
\item[DR] Division Ratio, коэффициент деления
\item[DSRC] Dedicated short-range communications, выделенная связь ближнего действия
\item[EDCA] Enhanced Distributed Channel Access, усовершенствованный доступ к каналу
\item[EDCF] Enhanced DCF, усовершенствованный DCF
\item[EM] Expectation Maximization, максимизация ожидания
\item[EPC] Electronic Product Code, электронный код продукции
\item[EPCID] Electronic Product Code Identifier, идентификатор, хранящийся в банке памяти EPC в метке
\item[HCCA] HCF Controlled Channel Access, доступ к каналу с использованием HCF
\item[HCF] Hybrid Coordinator Function, гибридная функция координации
\item[HTTP] HyperText Transfer Protocol, протокол передачи гипертекста
\item[IEC] International Electrotechnical Commission, Международная электротехническая комиссия
\item[IEEE] Institute of Electrical and Electronics Engineers, Институт инженеров электротехники и электроники
\item[ISO] International Organization for Standardization, международная организация по стандартизации
\item[IoE] Internet of Everything, интернет всего
\item[IoT] Internet of Things, интернет вещей
\item[JSON] JavaScript Object Notation, обозначение объектов JavaScript
\item[LLRP] Low Level Reader Protocol, протокол считывателя низкого уровня
\item[MAP] Markovian Arrival Process, марковский входящий поток
\item[MDID] Mask-Designer Identifier, идентификатор производителя метки
\item[MIB] Management Information Base, база информации управления
\item[MSP] Markov Service Process, марковский процесс обслуживания
\item[PC] Protocol Control, поле управления протоколом
\item[PH] Phase-Type Distribution, распределение фазового типа
\item[PIE] Pulse Interval Encoding, кодирование длительностью импульса
\item[PSK] Phase-Shift Keying, фазовая манипуляция
\item[QoS] Quality of Service, качество обслуживания
\item[REST] Representational state transfer, передача репрезентативного состояния
\item[RFID] Radio Frequency Identification, радиочастотная идентификация
\item[RPC] Remote Procedure Call, удаленный вызов процедур
\item[RTcal] Reader-Tag calibration, калибровочный символ для канала от считывателя к метке
\item[RTS] Request to Send, запрос на захват канала
\item[SaaS] Software as a Service, программное обеспечение как сервис
\item[SIFS] Short Interframe Space, короткий межкадровый интервал
\item[SIP] Session Initiation Protocol, протокол установления сеанса
\item[SNMP] Simple Network Management Protocol, простой протокол сетевого управления
\item[SNR] Signal-to-Noise Ration, соотношение сигнал-шум
\item[TCP] Transmission Control Protocol, протокол управления передачей
\item[TID] Tag Identification, банк идентификации метки
\item[TLS] Transport Layer Security,  протокол защиты транспортного уровня
\item[TRcal] Tag-Reader calibration, калибровочный символ для канала от метки к считывателю
\item[UDP] User Datagram Protocol, протокол пользовательских датаграмм
\item[UHF] Ultra High Frequency, ультравысокие частоты
\item[XaaS] Everything as a Service, что угодно как сервис
\item[XML] eXtensible Markup Language, расширяемый язык разметки
\item[ГИБДД] Государственная инспекция безопасности дорожного движения
\item[ЦКАД] Центральная кольцевая автодорога
\item[ЦОД] Центр обработки данных
\end{description}
