\chapter{Аналитическая модель системы радиочастотной идентификации автомобилей}\label{ch:ch3}


В предыдущей главе с помощью имитационной модели были получены результаты, показывающие, что периодическая смена значений флага инвентаризации сессии и переключение питания между антеннами приводят к увеличению количества раундов, в которых метка принимает участие (см. рис. \todo{Номер рисунка}{?}). Чтобы объяснить причину этого эффекта нужно учесть, что, согласно сделанным ранее допущениям, в случае обнаружения ошибки в передаче идентификатора меткой считыватель не передает ни повторные команды ACK, ни команды NACK, а переходит сразу к следующему слоту или раунду. Поэтому метка, не имея возможности узнать о произошедшей ошибке, инвертирует хранимое значение инвентаризационного флага текущей сессии (например, если флаг опроса в текущем раунде был равен $A$, то метка инвертирует флаг $A \rightarrow B$). Метка принимает участие в следующем раунде в одном из двух случаев: если считыватель сменил значение флага опроса (в нашем примере с $A$ на $B$), или если считыватель продолжает опрос со значением флага $A$, но перед этим сбросил питание на достаточно продолжительное время. В последнем случае не имеет значения, просто ли считыватель выключил передатчик, или же переключился на другую антенну -- лишь бы метка была выключена достаточно долго.

Имитационная модель, описанная ранее в предыдущей главе -- очень сложная, требующая значительного времени для получения статистически устойчивых результатов. Поэтому для более детального изчения этого эффекта была разработана аналитическая модель, основанная на неоднородных марковских случайных процессах. С помощью этой модели можно описать произвольный сценарий сбросов питания и переключения значений флагов сессий, учесть законы движения меток и оценить вероятность идентификации.

Аналитическая модель строится при ряде допущений. Для их четкого определения приводится описание модельного считывателя и меток, а после этого -- формальная постановка задачи, учитывающая введенные ограничения. Для описания работы считывателя вводится понятие сценария. Далее приводится описание раунда инвентаризации, вводятся формулы для оценки длительности раунда. Кроме того, приводится распределение вероятностей случайной величины, описывающей число меток, успешно передающих RN16 без коллизий. Далее опишем модель как пару случайных процессов, переходные вероятности которых зависят от сценария, позволяющих найти оценки длительностей отдельных раундов, и рассчитать вероятность идентификации метки. После этого приводятся выражения для расчета переходных вероятностей процессов, зависящих от заданого сценария. В конце приведем численные результаты и сделаем выводы.

Результаты, представленные в главе, были опубликованы в сборнике трудов конференции IEEE RFID'2018, индексируемом в Scopus и WoS \cite{RFID_IEEERFID2018}, а также представлены в трудах ВСПУ XIII \cite{RFID_VSPU2019}. Кроме того, доклады по полученным результатам были представлены на международных конференциях DCCN'2019 (Москва) и ICAAP\&SP'2019 (Коттайам, Индия).



% \section{Обзор литературы}
Протокол, лежащий в основе канального уровня стандарта EPC Gen-2 \cite{StdGen2}, "--- Framed Slotted ALOHA. Исходный протокол ALOHA был представлен в работе \cite{Abramson1970}. В этой же работе, на основе предположения о пуассоновском потоке передаваемых пакетов, была получена оценка пиковой пропускной способности сети как $1/(2e) \approx 0,186$. Позднее в работе \cite{Roberts1975} был описан способ повышения пропускной способности в два раза за счет разбиения времени на слоты (Slotted ALOHA).

Значительное число работ посвящено анализу производительности систем радиочастотной идентификации. Авторы пользуются эмпирическими методами \cite{Buettner2008}, используют аппарат марковских случайных процессов как с идеальным каналом \cite{Vogt2002, Wang2009, Vahedi2012, Vales-Alonso2009, Vales-Alonso2011, Tong2007, Vales-Alonso2017}, так и с каналом с ошибками \cite{DiMarco2014}, а также используют иные аналитические подходы для анализа производительности \cite{Ahmed2016, Yan2014, Jeon2009, Kim2007}. В ряде работ также сравнивается производительность протокола Frame Slotted ALOHA с альтернативными протоколами и расширениями \cite{Vahedi2014, LaPorta2011}.

Отметим работу \cite{Vales-Alonso2017}, в которой с помощью дискретной цепи Маркова исследуется эффект потери метками питания. Авторы описывают систему в виде двумерной цепи, моделируя изменение числа меток, которые продолжают участвоват в опросе, и число меток, уже передавших свои данные. В результате проведенного анализа, авторы находят оценки вероятности идентификации к концу раунда опроса и пропускную способность системы. В отличие от представленной в главе модели, потери пакетов возникают только вследствие коллизий, не учитываются сессии, и моделирование производится по слотами в одном раунде, а не по раундам. Также стоит выделить работу \cite{Pawowicz2020}, в которой идентификация транспорта с помощью UHF RFID рассматривается в контексте создания <<умных городов>>, и предлагается очень простая модель для оценки вероятности потери метки, то есть ее проезда области чтения без идентификации. Авторы делят область чтения на секции и, полагая, что секции имеют фиксированную длину, а передавшие свой идентификактор метки не участвуют в последующих раундах, предлагают простой метод расчета. Кроме того, в работе представлены результаты сравнения аналитических результатов и данных, полученных из стендового моделирования. Стоит отметить, что, как и в диссертационном исследовании, авторы \cite{Pawowicz2020} рассматривают изменения системы между раундами, в том числе "--- поступления и выходы метки из области чтения. В то же время, степень детализации модели в диссертационном исследовании существенно выше, используются более тонкие допущения.

Предложенная автором модель имеет ряд отличий. Во-первых, переходы марковской цепи происходят на границах раундов, а не слотов, и учитывают сбросы питания и смены флагов опроса меток. Во-вторых, модель учитывает потери произвольных пакетов из-за ненулевой битовой ошибки. В-третьих, модель состоит из двух марковский процессов, первый из которых нужен для расчета числа меток, участвующих в раундах, а второй "--- для оценки вероятности идентификации. В-четвертых, матрицы переходных вероятностей строятся в виде произведения матриц отдельных операций (проведение опроса, инвертирование флага опроса, сброс питания, добавление или удаление метки из области чтения). Из-за этого, с одной стороны, марковские процессы оказываются неоднородными и тяжелее в анализе, но, с другой стороны, делает предложенный метод очень гибким, и позволяет в дальнейшем расширить его на более сложные операции.





%%%%%%%%%%%%%%%%%%%%%%%%%%%%%%%%%%%%%%%%%%%%%%%%%%%%%%%%%%%%%%%%%%%%%%%%%%%%%%%%
\section{Ограничения и допущения}\label{sec:ch3_assumptions}
%%%%%%%%%%%%%%%%%%%%%%%%%%%%%%%%%%%%%%%%%%%%%%%%%%%%%%%%%%%%%%%%%%%%%%%%%%%%%%%%
Для построения аналитической модели введём ряд необходимых допущений и ограничений на считыватель и метки, а также опишем приближенный способ расчета распределения длительностей раундов.


%%% --------------------------------------------
\subsection{Модельный считыватель}
%%% --------------------------------------------
В определении \textbf{модельного считывателя} мы накладываем ограничения на область чтения, битовые ошибки в принятых ответах меток, длительности команд, схему раунда инвентаризации. Наиболее существенные ограничения касаются области чтения и битовых ошибок.

Будем считать, что метки движутся по прямой. \textbf{Областью чтения} модельного считывателя будем считать фиксированный интервал, в каждой точке которого метка имеет достаточно энергии для включения и приема команд от считывателя. Вне области чтения метки не работают, с единственным исключением: если метка, покидающая область чтения, уже принимает участие в раунде инвентаризации, считается, что она сможет передать свои ответы в любом из слотов.

Если при передаче ответа от метки произошла коллизия с ответом другой метки, считыватель не может принять ни один из ответов. Если же при передаче ответа коллизии не произошло, вероятность \textbf{битовой ошибки (BER)} в принятом ответе постоянна\footnote{Это наиболее существенное ограничение, которое позволяет игнорировать точное расположение меток в области чтения, а также изменчивость канала во времени из-за эффекта Доплера.} и равна $\beta$. Возникновение ошибок в каждом бите будем полагать независимым, т.е. вероятность успешного приёма $b$ бит от одной или нескольких меток есть $b^{1 - \beta}$.

Модельный считыватель не изменяет значения параметров Q, Tari, M, TRext, а также калибровочных интвервалов TRcal и RTcal. Число Q задает двоичный логарифм числа слотов, то есть число слотов $N_s = 2^Q$. Значение Tari "--- это длительность символа data-0, равная 6,25, 12,5, 18,75 или 25,0~мкс. Значение M задает способ кодирования ответов меток, оно принимает значение 0 для кодирования FM0 и 1, 2, или 3 для кодов Миллера длиной 2, 4 или 8 символов соответственно. TRext "--- это флаг, задание которого означает, что метка должна использовать расширенную преамбулу перед своими ответами. Наконец, калибровочные интервалы TRcal и RTcal нужны для корректного разделения нулей и единиц, и для вычисления меткой скорости передачи ответов. Также считыватель всегда использует одну и ту же сессию для опроса меток; для определенности и простоты будем считать, что считыватель всегда использует сессию S0.

\textbf{Длительности команд} будем считать фиксированными, игнорируя вариации, связанные с использованием кодировки PIE и полями, неизвестными заранее, "--- например, случайным числом RN16 в команде ACK или значением контрольных сумм. Длительности команд будем обозначать как $T\{\text{msg}\}$, где $\text{msg}$ -- название команды. Интервалы\footnote{Диапазоны интервалов $T_1, T_2 \text{ и } T_3$ определены в табл. 6.16 стандарта EPC Gen.2. В тексте используются стандартные обозначения этих интервалов; в тех случаях, когда смысл $T_i$ может быть неверно истолкован, будет приведено дополнительное пояснение.} времени между концом команды и началом ответа метки ($T_1$), между концом ответа метки и началом следующей команды ($T_2$), а также между последовательными командами, если ответа от метки не было ($T_1 + T_3$), считаются постоянными.

Считыватель не использует команды Select, NACK, QueryAdjust, а также повторные команды ACK. Для чтения банков памяти используется только команда Read, передаваемая после получения ответа на команду Req\_RN. Команда Read всегда направлена на один и тот же банк памяти и запрашивает чтение одного и того же числа слов.

В каждый момент модельный считыватель может выполнять одну из \textbf{трех операций}: производить опрос меток в раунде инвентаризации, сбрасывать питание или изменять флаг опроса. Операции выполняются последовательно, ненулевую длительность имеют только операции инвентаризации и сброса питания.

\begin{enumerate}
	\item \textbf{Операция сброса питания} занимает время $T_\Delta$, на это время считыватель отключает питание. Отключение и последующее включение питания происходят мгновенно. После включения считыватель готов сразу начать инвентаризацию с флагом Target, равным A.

	\item \textbf{Операция смены флага инвентаризации} осуществляется мгновенно, в результате её выполнения инвертируется флаг сессии, по которому производится опрос меток (поле Target команды Query).

	\item \textbf{Операция инвентаризации} заключается в выполнении раунда опроса меток, начинается с передачи команды Query и завершается вместе с последним тактом.
\end{enumerate}

При оценке длительности раундов инвентаризации будем считать время распространения сигналов константой, равной $\delta$.


%%% --------------------------------------------
\subsection{Модельные метки}
%%% --------------------------------------------
Будем считать, что любой \textbf{модельной метке} всегда достаточно энергии в области чтения для включения и успешного приёма команд считывателя без ошибок. Также предполагаем, что метка при включении всегда сбрасывает значение хранимого флага в $A$, то есть длительность хранения флага всегда меньше, чем время отключения считывателя. Модельные метки не учитывают содержимое полей PC и слов XPC\_W1 и XPC\_W2 при формировании ответов на команды ACK. Для определенности будем предполагать, что в этих ответах метки передают PC, EPC и CRC. Длину EPCID считаем одинаковой для всех меток.

\textbf{Длительности ответов меток} будем считать известными и фиксированными. Будем обозначать их так же, как и длительности команд: $T_\text{msg}$, где $\text{msg}$ "--- название ответа. Размеры ответов в битах будем обозначать как $|\text{msg}|_b$.

Также будем считать, что метки не используют паролей, считывателю не нужно передавать команду Access при чтении банков памяти. Команда Kill не рассматривается, и метка не может быть в <<убитом>> состоянии, когда при включении она сразу попадает в состояние KILLED и не отвечает на любые команды.




%%%%%%%%%%%%%%%%%%%%%%%%%%%%%%%%%%%%%%%%%%%%%%%%%%%%%%%%%%%%%%%%%%%%%%%%%%%%%%%%
\section{Постановка задачи}\label{sec:ch3_problem_definition}
%%%%%%%%%%%%%%%%%%%%%%%%%%%%%%%%%%%%%%%%%%%%%%%%%%%%%%%%%%%%%%%%%%%%%%%%%%%%%%%%
Будем считать, что метки движутся по одной прямой, а область чтения, в которой метки получают достаточно энергии для работы, является интервалом $(0, L)$. Пусть функция $x(t)$ определяет закон движения меток так, что $x(0) = 0,\; \exists\,T_L: x(T_L) = L$ и $\forall\, t \in (0, T_L): x(t) \in (0, L)$, а $\forall t > T_L: x(t) > L$. Метки появляются в известные моменты времени $a_1, a_2, a_3, \dots$ в точке $x=0$, их движение определяется функциями $x_i(t) \equiv x(t - a_i)$. Другими словами, метки движутся одинаково со сдвигом по времени, могут ускоряться, замедляться или моментально изменять свое местоположение, но после первого покидания области чтения, которое обязательно должно произойти, более никогда не возвращаются в неё.

Так как законы движения меток $x_i(t)$ известны и детерминированы, число меток в области чтения в момент $t$ можно определить как функцию $N(t) = |\{ i:\: a_i \leqslant t \leqslant a_i + T_L \}|$. Для корректного определения случайных процессов потребуем, чтобы эта функция была ограничена сверху, то есть $\exists \overline{N} = \max\limits_{t \geqslant 0} N(t)$. Для этого достаточно потребовать, чтобы существовал интервал $\Delta t$, такой, что $\forall i: a_{i+1} - a_i \geqslant \Delta t$.

Благодаря допущению о постоянстве BER при приеме ответов от меток, в дальнейшем будет гораздо проще построить модель, так как единственной существенной величиной, характеризующей множество меток, будет функция $N(t)$, а не отдельные значения координат $x_i(t)$ или скоростей $x'_i(t)$ меток. Хотя это допущение является очень сильным, оно может быть более или менее существенно в зависимости от взаимного положения меток и считывателя. Например, при движении меток по горизонтали на конвейерной ленте и горизонтальном размещении антенны считывателя над лентой уровень сигнала будет более равномерным, чем при размещении меток на машинах, а считывателя "--- под углом к дороге.

Последнее допущение относительно области чтения заключается в том, что метки покидают область чтения только на границе раундов. Другими словами, если в раунде приняло участие $n$ меток, то любая метка остается в области чтения до конца и может ответить считывателю в любом из слотов.

Пусть $r \in \mathbb{N}$ -- произвольный номер раунда инвентаризации. Поведение считывателя характеризуется значением опрашиваемого флага $X_r \in \{A,B\}$ и признаком сброса питания после раунда $e_r \in \{0,1\}$.

\begin{defn}
	\textit{Спецификацией раунда} будем называть пару значений опрашиваемого флага и признака сброса питания $(X, e)$, которую будем сокращенно обозначать символом $\alpha \defeq X^{e}$.
\end{defn}

Например, $\alpha_5 = B^{1}$ означает, что в пятом раунде считыватель ведет опрос по флагу инвентаризации $B$ и после раунда сбрасывает питание, а $\alpha_2 = A^{0}$ говорит о том, что во втором раунде опрос идет по флагу $A$ и питание в конце не сбрасывается.

\begin{defn}\label{def:ch3_reader_scenario}
	\textit{Сценарием работы считывателя} $\bm{\alpha} = \alpha_1 \alpha_2 \dots \alpha_R$ будем называть последовательность спецификаций раундов конечной длины $R$.
\end{defn}

Будем считать, что после $R$-го раунда считыватель начинает <<проигрывать>> сценарий с самого начала, с первого раунда. Например, переключение флага каждые два раунда со сбросом питания на каждом четвертом можно описать как сценарий $A^0, A^0, B^0, B^1$.

Под \textit{идентификацией метки} мы будем понимать, как и в ранее во второй главе, либо успешную передачу EPCID метки, либо передачу TID.


\begin{probl}\label{problem:ch3_id_prob}
	Пусть известны законы движения меток $x_i(t) \equiv x(t - t_i)$, вероятность битовой ошибки в передаче ответов $\beta$ и сценарий работы считывателя $\bm{\alpha} = \alpha_1 \alpha_2 \dots \alpha_R$, а также размеры и длительности команд считывателя и ответов меток. Пусть также для идентификации меток требуется только EPCID (или комбинация EPCID и TID). Требуется найти вероятность, с которой каждая метка будет успешно идентифицирована.
\end{probl}

Для решения этой задачи опишем два случайных процесса: \textit{фоновый процесс}, моделирующий число меток, участвующих в каждом раунде, и \textit{основной процесс}, моделирующий проезд области чтения отдельно взятой меткой и позволяющий оценить вероятность её идентификации. Для описания переходов основного процесса необходимо иметь оценку длительностей раундов, которую можно вычислить, исходя из оценки числа участвующих в каждом раунде меток. Последнюю оценку получим с помощью итерационного расчета стационарного распределения фонового процесса.


%%%%%%%%%%%%%%%%%%%%%%%%%%%%%%%%%%%%%%%%%%%%%%%%%%%%%%%%%%%%%%%%%%%%%%%%%%%%%%%%
\section{Моделирование раундов инвентаризации}\label{sec:ch3_inventory}
%%%%%%%%%%%%%%%%%%%%%%%%%%%%%%%%%%%%%%%%%%%%%%%%%%%%%%%%%%%%%%%%%%%%%%%%%%%%%%%%
Прежде, чем перейти к построению пространства состояний и операций над ним, рассмотрим подробнее процесс идентификации меток и найдем оценки следующих величин:

\begin{itemize}
	\item вероятность $P_n(m)$ передачи EPCID $m$ метками из $n$;
	\item вероятность $p_\text{id}$ успешной идентификации отдельной метки;
	\item средняя ожидаемая длительность $\tau = \tau(n)$ раунда с $n$ метками;
	\item максимальная длительность раунда $\tau_{max}$.
\end{itemize}

\begin{figure}[htb]
	\centerfloat{
		\includegraphics[width=1.0\textwidth]{chapter3/ch3_inventory_round}
  }
  \legend{Справа вверху "--- пример слота, в котором произошла ошибка в передаче ответа Handle. Справа внизу "--- пример слота, в котором метка успешно передает идентификатор, если идентификация происходит только по EPCID.}
  \caption[Пример раунда опроса при идентификации по EPCID и TID.]{Пример раунда инвентаризации с $Q=2$ и числом слотов $N_s=2^Q=4$ в случае идентификации метки по EPCID и TID. }
  \label{fig:ch3_inventory_round}
\end{figure}

Пример раунда инвентаризации показан на рис.~\ref{fig:ch3_inventory_round}. В этом примере предполагается, что параметр $Q = 2$, то есть число слотов $N_s = 2^Q = 4$, и в раунде участвует три метки. Первая метка выбрала для передачи первый слот, а вторая и третья -- третий. Идентификация метки происходит по комбинации EPCID и TID, поэтому в первом слоте считыватель после получения EPCID запрашивает случайное слово RN16 командой Req\_RN и использует его для контроля доставки в команде Read. В ответ на последнюю метка передает содержимое своего банка памяти TID. Если бы идентификация происходила только по EPCID, то первый слот был бы проще, обмен сообщениями выглядел бы как на правом нижнем фрагменте. Справа вверху показан пример слота, в котором произошла ошибка в передаче одного из ответов. Так как модельный считыватель не передает команды повторно, слот на этом завершается, метка остается неидентифицированной.

Ошибка может произойти при передаче любого из ответов. Учитывая то, что BER постоянен и равен $\beta$, вероятность успешной передачи ответа $\text{msg}$ можно найти как:

\begin{equation}\label{eq:ch3_response_err}
	P_\text{rx}\{\text{msg}\} = (1 - \beta)^{|\text{msg}|},
\end{equation}
где $|\text{msg}|$ -- длина ответа $\text{msg}$ в битах.

В дальнейшем потребуется оценивать число меток, участвующих в очередном раунде. Введём следующее определение.

\begin{defn}
	Назовем метку \textit{активной}, если она находится в области чтения, и ее значение флага сессии совпадает с передаваемым считывателем в поле Target команды Query. Значение этого поля будем называть \textit{флагом опроса}.
\end{defn}
\begin{rem}
	Следует подчеркнуть, что все модельные метки являются пассивными в техническом смысле слова, то есть они работают только в поле действия считывателя и не содержат автономного источника питания. Данное определение активности "--- исключительно логическое. Оно было выбрано, так как наиболее точно и лаконично описывает состояние метки.
\end{rem}

Обозначим $\mu_0$ "--- число пустых слотов в раунде, $\mu_1$ "--- число слотов с ответом единственной метки без коллизий и $\mu_2$ "--- число слотов с коллизиями. Для дальнейших расчетов нам потребуются распределения вероятностей случайных величин $\mu_0$ и $\mu_1$. Распределение $\mu_0$ определяется следующим образом:

\begin{equation}\label{eq:ch3_empty_slots}
	\mathbb{P}\{\mu_0 = z\} = \frac{1}{N_s^n} C_{N_s}^z {n\brace N_s-z} (N_s - z)!,
\end{equation}
где ${ n\brace N_s-z }$ "--- число Стирлинга 2-го рода, то есть число способов разбиения $n$ меток по $N_s - z$ непустым подмножествам.

Вероятность события $\{ \mu_1 = m \}$, то есть того, что ровно $m$ меток выбрали уникальные слоты и ответят считывателю без коллизий, будем обозначать как $\overline{P}_n(m) = \mathbb{P}\{ \mu_1 = m \}$, и будем вычислять с помощью формулы, приведенной в работе \cite{Vales-Alonso2011}:

\begin{equation}\label{eq:ch3_nocol_pnm}
	\overline{P}_n(m) = \mathbb{P}\{ \mu_1 = m \} = \frac{N_s! n!}{m! N_s^n} \sum\limits_{z=0}^{n-m}
		\frac{(-1)^z (N_s - m - z)^{(n - m - z)}}{(n - m - z)! z! (N_s - m - z)!}
\end{equation}

Сделаем два замечания относительно формул \eqref{eq:ch3_empty_slots} и \eqref{eq:ch3_nocol_pnm}. Во-первых, в \eqref{eq:ch3_nocol_pnm} предполагается, что число активных меток $n$ не превосходит числа слотов $N_s$. Во-вторых, на практике обе формулы неудобно применять из-за очень больших чисел в числителях и знаменателях уже при $N_s > 8$, то есть $Q = \log_2 N_s > 3$. Частично эту проблему удается решить группировкой множителей, но не во всех случаях, поэтому при численных расчетах для вычисления обеих вероятностей будем пользоваться методом Монте-Карло.

\begin{prop}\label{prop:ch3_pnm}
	Если в раунде инвентаризации участвует $n > 0$ меток, $P_\text{rx}\{\text{RN16}\}$ -- вероятность успешной передачи ответа RN16, а $\overline{P}_n(k)$ -- вероятность того, что ровно $k \leqslant n$ меток выбрали такие слоты, которые не выбрали другие метки, то вероятность того, что ровно $m \leqslant n$ меток изменят после раунда хранимое значение флага определяется следующим выражением:
	\begin{equation}\label{eq:ch3_pnm}
		P_n(m) = \sum\limits_{i=m}^{n}\overline{P}_n(i) C_i^mx \left(P_\text{rx}\{\text{RN16}\}\right)^m
			\left(1 - P_\text{rx}\{\text{RN16}\}\right)^{i - m},\; m \leqslant n
	\end{equation}
\end{prop}
\begin{proof}
	Метка передает идентификатор, если ей пришла команда ACK со случайным словом, совпадающим с переданным ранее меткой в ответе RN16. Согласно сделанным предположениям, это происходит, когда метка единственной передавала своё случайное слово RN16 в слоте, без коллизий с другими метками, и это слово было успешно доставлено\footnote{Если бы использовалась более точная модель интерференции, получение ответа было бы возможно также, например, если сигнал от другой меxтки был в том же слоте, но очень ослабленный.}. Тогда событие $X$:

	$$
	X = \text{<<ровно $m$ из $n$ меток передали идентификаторы>>}
	$$
	можно представить в виде суммы непересекающихся событий $X = \sum\limits_{i=m}^n Y_i$, где:
	$$
	\begin{aligned}
	Y_i = &\text{<<ровно $i$ из $n$ меток передали RN16 без коллизий,}\\
		  &\text{и ровно $m$ из $i$ из них было принято без ошибок>>}.
	\end{aligned}
	$$

	По формуле полной вероятности, учитывая независимость выбора слотов метками и возникновения ошибок в передаче их ответов, а также независимость возникновения ошибок в каждом переданном метками бите, получаем цепочку равенств, доказывающих утверждение:

	\begin{align*}
		P_n(m) &= \mathbb{P}\{X\} = \sum\limits_{i=m}^{n} \mathbb{P}\{Y_i\}\\
		&\begin{aligned}= \sum\limits_{i=m}^{n}(
			&\mathbb{P}\{\text{ровно $i$ из $n$ меток выбрали свободные слоты}\} \times\\
			& \mathbb{P}\{\text{ровно $m$ из $i$ меток успешно передали RN16}\})
			\end{aligned}\\
		&=\sum\limits_{i=m}^{n} \overline{P}_n(i) \times \left(C_i^m(P_\text{rx}\{\text{RN16}\})^m
			(1 - P_\text{rx}\{\text{RN16}\})^{i - m}\right)
	\end{align*}

\end{proof}

Для того, чтобы удостовериться, что $P_n(m)$ корректно описывает вероятность, докажем, что $\sum\limits_{m = 0}^n P_n(m) = \sum\limits_{i=0}^n \overline{P}_n(i) = 1$ (справедливость последнего равенства утверждается в работе \cite{Vales-Alonso2011}). Действительно,
$$
\begin{aligned}
	\sum\limits_{m=0}^n P_n(m) &=
		\sum\limits_{m=0}^n \sum\limits_{i=m}^n \overline{P}_n(i) C_i^m
			P_\text{rx}\{\text{RN16}\}^m (1 - P_\text{rx}\{\text{RN16}\})^{i-m}\\
	&= \sum\limits_{i=0}^n \sum\limits_{m=0}^i \overline{P}_n(i) C_i^m
			P_\text{rx}\{\text{RN16}\}^m (1 - P_\text{rx}\{\text{RN16}\})^{i-m}\\
	&= \sum\limits_{i=0}^n \overline{P}_n(i) \sum\limits_{m=0}^i C_i^m
			P_\text{rx}\{\text{RN16}\}^m (1 - P_\text{rx}\{\text{RN16}\})^{i-m}\\
	&= \sum\limits_{i=0}^n \overline{P}_n(i) \left( P_\text{rx}\{\text{RN16}\} + (1 - P_\text{rx}\{\text{RN16}\}) \right)^i\\
	&= \sum\limits_{i=0}^n \overline{P}_n(i)
\end{aligned}
$$

Вероятность успешной идентификации зависит от того, что используется в качестве идентификатора: EPCID или комбинация TID и EPCID. Условную вероятность успешной идентификации метки при условии отсутствия коллизии при её передаче будем обозначать как $p_{\text{id}}$:

\begin{equation}\label{eq:ch3_p_id}
	p_{\text{id}} = \begin{cases}
		P_{\text{rx}}\{\text{EPCID}\}, &\text{только EPCID}\\
		P_{\text{rx}}\{\text{EPCID}\}P_{\text{rx}}\{\text{Handle}\}P_{\text{rx}}\{\text{TID}\},&\text{EPCID и TID}
	\end{cases}
\end{equation}

Для вычисления оценки средней длительности раунда, воспользуемся приближенной схемой расчёта в допущении о независимости определения типов слотов. Будем считать, что каждый слот может иметь один из трёх типов: пустой слот (без ответа метки), слот с коллизией (ошибка при передаче RN16) и слот с ответом без коллизии. Обозначим через $t_0$ длительность пустого слота, $t_1^{\text{(epc)}}$ и $t_1^{\text{(tid)}}$  "--- средние длительности слотов с ответом без коллизий при идентификации по EPCID или по TID соответственно, а через $t_2$ "--- длительность слота с коллизией. Эти длительности могут быть вычислены следующим образом:

\begin{equation}\label{eq:ch3_slot_durations}
	\begin{aligned}
		t_0 =\;& T_\text{QRep} + T_1 + T_3\\
		t_1^{\text{(epc)}} =\;& T_\text{QRep} + (T_1 + T_2) +
			T_\text{RN16} + P_{rx}\{\text{RN16}\} \times \\
			&\left( T_\text{ACK} + (T_1 + T_2) + T_\text{EPCID} \right)\\
		t_1^{\text{(tid)}} =\;& T_\text{QRep} + (T_1 + T_2) +
				T_\text{RN16} + P_{rx}\{\text{RN16}\} \times \\
			&( T_\text{ACK} + (T_1 + T_2) + T_\text{EPCID} + P_{rx}\{\text{EPCID}\} \times \\
			&\quad (T_\text{Req\_RN} + (T_1 + T_2) + T_\text{Handle} 	+ P_{rx}\{\text{Handle}\} \times \\
			&\quad\quad(T_\text{Read} +(T_1 + T_2) + T_\text{TID})))\\
		t_1 =\;& \begin{cases}
			t_1^{\text{(epc)}}, &\text{только \text{EPCID}}\\
			t_1^{\text{(tid)}}, &\text{\text{EPCID} и \text{TID}}
 		\end{cases}\\
 		t_2 =\;& T_\text{QRep} + (T_1 + T_2) + T_\text{RN16}
	\end{aligned}
\end{equation}

Согласно допущению, вероятности слотов определяются независимо. Обозначим $p_0(n)$, $p_1(n)$ и $p_2(n)$ веротности того, что произвольный слот пуст, содержит ответ ровно одной метки или коллизию, соответственно. Эти вероятности можно вычислить, используя \eqref{eq:ch3_empty_slots} и \eqref{eq:ch3_nocol_pnm}:

\begin{equation}\label{eq:ch3_slot_probs}
	\begin{aligned}
		p_0(n) =\;& \frac{1}{N_s} \mathbb{E} \mu_0 = \frac{1}{N_s} \sum\limits_{i=0}^{N_s} i \mathbb{P}\{\mu_0 = i\}\\
		p_1(n) =\;& \frac{1}{N_s} \mathbb{E} \mu_1 = \frac{1}{N_s} \sum\limits_{i=0}^{N_s} i \mathbb{P}\{\mu_1 = i\}\\
		p_2(n) =\;& 1 - p_0(n) - p_1(n)
	\end{aligned}
\end{equation}

С помощью соотношений \eqref{eq:ch3_slot_durations} и \eqref{eq:ch3_slot_probs} длительность раунда $\tau(n)$ можно определить как сумму математического ожидания длительности слота и, если после раунда сбрасывается питание, длительности этого сброса:

\begin{equation}\label{eq:ch3_round_duration_of_n}
	\tau_r(n) = N_s \sum\limits_{i=0}^{2}p_i(n)t_i +
		(T_\text{Query} - T_\text{QRep}) +
		e_r T_\downarrow,
\end{equation}
где $n$ "--- число меток, участвующих в раунде, $e_r$ "--- индикатор сброса питания после опроса, а $T_\downarrow$ "--- длительность сброса. Слагаемое $(T_\text{Query} - T_\text{QRep})$ необходимо, так как при расчете длин слотов $t_i$ предполагалось, что каждый слот начинается с команды QRep, хотя первый слот начинается с более длинной команды Query.

Хотя в действительности распределение величин $\mu_0$, $\mu_1$ и $\mu_2$ не являются независимыми, приведенный способ расчета очень простой и позволяет получить достаточно точную оценку длительности раундов. Так, на рис.~\ref{fig:ch3_round_durations_validation} показан расчет с помощью описанной аналитической модели и метода Монте-Карло при различных значениях BER и числе участвующих в опросе меток от 0 до 20 (остальные параметры были фиксированы, число слотов $N_s = 2^Q = 16$). Можно видеть, что допущение о независимости определения типа слота оказывает минимальное влияние на точность оценки.

\begin{figure}[htb]
	\centerfloat{
		\includegraphics[width=1.0\textwidth]{chapter3/ch3_round_durations_validation.pdf}
  }
  \caption[Валидация модели расчета средрней длительности раундов.]{Сравнение результатов расчета средней длительности раундов с помощью аналитической модели и метода Монте-Карло.}
  \label{fig:ch3_round_durations_validation}
\end{figure}


Если известно распределение вероятностей числа меток, участвующих в раунде, $\bm{\pi} \in \mathbb{R}^{\overline{N}+1}$, где $\pi_i$ "--- вероятность того, что в системе ровно $i$ активных меток, $i \in [0, \overline{N}]$, то среднюю длину раунда можно опередлить как математическое ожидание $\tau_r = \mathbb{E} \tau_r(n)$:

\begin{equation}\label{eq:ch3_round_duration_avg}
	\tau_r = \sum\limits_{i=0}^{\overline{N}} \pi_i \tau_r(i)
\end{equation}

Пользуясь выражениями для длительностей слотов $t_1, t_2$ и $t_3$, можно найти оценку максимальной длительности раунда $\tau_{max}$:

\begin{equation}\label{eq:ch3_max_round_duration}
	\tau_{max} = \begin{cases}
 		\overline{N} t_1 + (N_s - \overline{N}) t_0, &\overline{N} \leqslant N_s\\
 		(N_s - 1) t_1 + t_2, &\overline{N} > N_s.
 	\end{cases}
\end{equation}
Справедливость этого выражения следует из того, что $t_0 < t_2 < t_1$ (значение $T_3$ заведомо не превышает $T_\text{RN16}$): если слотов больше, чем меток, то максимум достигается, когда все метки успешно передают свои идентификаторы. Если же слотов меньше, то в самом длинном раунде все слоты кроме одного будут содержать успешные передачи, а в одном слоте произвойдет коллизия оставшихся $\overline{N} - (N_s - 1)$ меток.



%%%%%%%%%%%%%%%%%%%%%%%%%%%%%%%%%%%%%%%%%%%%%%%%%%%%%%%%%%%%%%%%%%%%%%%%%%%%%%%%
\section{Вычисление оценки длительностей раундов}\label{sec:ch3_round_durations}
%%%%%%%%%%%%%%%%%%%%%%%%%%%%%%%%%%%%%%%%%%%%%%%%%%%%%%%%%%%%%%%%%%%%%%%%%%%%%%%%

%%% --------------------------------------------
\subsection{Размеченные сценарии и элементарные операции}
%%% --------------------------------------------
Пусть в некоторый момент времени в системе находится $N$ меток, $0 \leqslant N \leqslant \overline{N}$, и пусть $n$ из них активны, то есть готовы принять участие в раунде, в котором опрос просходит по флагу $X \in \{A, B\}$, $0 \leqslant n \leqslant N$.

Число $n$ может измениться по трем причинам (см. рис.~\ref{fig:ch3_operations}). Во-первых, если флаг опроса $X$ после очередного раунда инвентаризации не изменяется, и $n' \leqslant n$ меток в этом раунде передали свои EPCID, то эти метки инвертировали свои флаги и в следующем раунде не будут принимать участие, то есть активными будут $n - n'$ меток. Если же после раунда считыватель инвертирует флаг опроса $X$, то число активных меток будет $N - (n - n')$. Во-вторых, само число $N$ может измениться, если какая-либо метка покинула систему, или, наоборот, вошла в нее. Это изменение может затронуть число активных меток $n$, если систему покинула активная метка (в этом случае $n$ уменьшается на единицу), или если метка поступила в систему, и текущий флаг опроса равен $A$ (значение $n$ увеличивается на единицу). Наконец, в-третьих, считыватель может сбросить питание, после чего все метки будут хранить значение флага, равное $A$. Если считыватель продолжит опрос с флагом $X = A$, то активными будут все $n = N$ меток, а если с флагом $B$, то активных меток точно не будет, $n = 0$.

\begin{figure}[htb]
	\centerfloat{
    \includegraphics[width=1.0\textwidth]{chapter3/ch3_operations}
  }
  \caption{Операции над системой.}
  \label{fig:ch3_operations}
\end{figure}

Таким образом, изменение числа активных меток происходит в результате пяти действий, которые в дальнейшем будем называть \textit{элементарными операциями}: проведения раунда инвентаризации ($U_N^\nabla$), изменения флага опроса ($U_N^\times$), сброса питания ($U_N^\downarrow$), добавления метки в систему ($U_N^+$) или выхода метки ($U_N^-$). Как будет показано далее, с помощью композиции элементарных операций можно описать, как изменяется распределение числа активных меток после каждого раунда. Однако первый вопрос "--- как по известному сценарию работы считывателя определить, какие элементарные операции нужно выполнять.

В начале главы было введено понятие сценария (см. определение~\ref{def:ch3_reader_scenario}) как последовательности $\bm{\alpha} = \alpha_1 \alpha_2 \dots \alpha_R$, где $\alpha_r = (X_r, e_r) \equiv X_r^{e_r}$ "--- спецификация раунда. По заданному сценарию можно описать операцию, которую считыватель должен провести над метками. Например, если $\bm{\alpha} = \alpha_1 \alpha_2 \alpha_3 \alpha_4 = A^0 B^1 A^0 A^1$, то в первом раунде нужно провести опрос ($U_N^\nabla$) и инвертировать флаг опроса с $A$ на $B$ ($U_N^\times$), во втором раунде "--- провести опрос и сбросить питание ($U_N^\downarrow$), в третьем раунде просто провести опрос, а в четвертом "--- провести опрос и сбросить питание.

В сценарии содержится только информация о работе считывателя, однако нет данных о том, когда метки появляются и выходят из системы. Предположим, что каким-то образом удалось установить, сколько меток содержится в системе в начале каждого раунда, и обозначим последовательность этих значений как $\{ N_r \}_{r=1}^R$. Введем следующее определение, объединяющее сценарий $\bm{\alpha}$ и оценки $\{ N_r \}$.

\begin{defn}\label{ref:ch3_marked_scenario}
  Будем называть \textit{размеченным сценарием} последовательность символов $\widetilde{\bm{\alpha}} = \widetilde{\alpha}_1 \widetilde{\alpha}_2 \dots \widetilde{\alpha}_R$, каждый из которых имеет вид тройки $\alpha_r = (X_r, e_r, N_r)$. Символы $\alpha_r$ будем называть \textit{размеченными спецификациями раундов}. Для сокращения записи будем также обозначать размеченные спецификации как $\widetilde{\alpha}_r = X_{N_r}^{e_r}$.
\end{defn}

Рразмеченный сценарий получаются из обычного сценария $\bm{\alpha}$ добавлением нижнего индекса из $\{ N_r \}$ к каждому символу $\alpha_r = X^{e_r}$, задающему спецификацию $r$-го раунда. Возвращаясь к предыдущему примеру, если сценарий $\bm{\alpha} = A^0 B^1 A^0 A^1$ и $\{ N_r \} = \{ 4, 5, 4, 3 \}$, то $\widetilde{\bm{\alpha}} = A^0_4 B^1_5 A^0_4 A^1_3$. Для такого размеченного сценария уже можно полностью описать все действия над системой: в первом раунде после опроса ($U_4^\nabla$) нужно инвертировать флаг ($U_4^\times$) и добавить метку ($U_{B,4}^+$), во втором раунде "--- провести опрос ($U_5^\nabla$), сбросить питание ($U_5^\downarrow$) и удалить одну метку ($U_5^-$), в третьем "--- провести опрос ($U_4^\nabla$) и удалить еще одну метку ($U_4^-$), а в четвертом "--- провести опрос ($U_3^\nabla$), сбросить питание ($U_3^\downarrow$) и добавить метку ($U_{A_3}^+$).

Ддля построения размеченного сценария необходимо найти последовательность $\{ N_r \}$. Обозначим последовательность моментов начала раундов опросов как $\{ t_r \}_{r=1}^\infty$. Для искомой последовательности выполняется равенство $N_r = f_N(t_r)$. Таким образом, для нахождения $N_r$ нужно знать, когда начинается каждый раунд, а для этого нужно вычислить оценки длительностей каждого раунда. Формально, если известна последовательность длительностей раундов $\{ \tau_r \}_{r=1}^\infty$, то значение $N_r$ можно определить как:
$$
N_r = f_N(\tau_1 + \tau_2 + \dots + \tau_{r-1}).
$$
Длительности раундов опроса $\{ \tau_r \}$ вычисляются с помощью выражений \eqref{eq:ch3_round_duration_of_n} и \eqref{eq:ch3_round_duration_avg}.

Прежде, чем идти дальше, сделаем одно замечание. При определении закона изменения числа меток в системе $f_N(t)$ было сделано допущение о том, что между моментами изменения числа меток в системе $\{ t_i^a \}$ есть минимальный интервал $\underline{\Delta t}$, причем максимальная длительность раунда $\tau_{max} < \underline{\Delta t}$. Благодаря этому допущение нет необходимости рассматривать случаи, когда число меток в соседних раундах отличается более, чем на единицу, а также случай, когда общее число меток не менялось, но одна метка успела покинуть систему, а другая "--- поступить. Снять это допущение легко: достаточно при вычислении $\{ N_r \}$ дополнительно учитывать, сколько меток поступает в систему течение раунда ($\Delta N_r^+$), а сколько "--- выходит ($\Delta N_r^-$). Так как $N_{r+1} = N_r + \Delta N_r^+ - \Delta N_r^-$, то явно указывать значения $N_r$, кроме $N_1$, не нужно, и вместо добавления нижнего индекса $N_r$ в размеченных спецификациях раундов можно было бы добавлять $\Delta N_r^+$ и $\Delta N_r^-$. С соответствующими поправками, все остальное, описанное в этой главе, остается справедливым для такого более общего случая. Однако, это значительно усложнит обозначения и описание операций, поэтому ограничимся более простым случаем. Кроме того, в исследуемой системе интервалы между поступлениями меток существенно больше, чем длительности раундов, и предположение о существовании $\underline{\Delta t}$ выглядит вполне естественным.

Далее рассмотрим, как формально описать элементарные операции. Затем опишем построение операций, моделирующих проведение раундов согласно заданному сценарию и оценкам числа меток в каждом раунде $\{N_r\}$. После этого приведем схему расчета распределения числа активных меток, а в завершение раздела "--- итерационный алгоритм расчета длительностей раундов.


%%% --------------------------------------------
\subsection{Матрицы элементарных операций}\label{subsec:ch3_bg_elem_op_matrices}
%%% --------------------------------------------
Как отмечалось ранее, число активных меток является случайной величиной, так как в общем случае вероятность того, что в течение раунда инвентаризации ровно $n - n'$ меток передадут свои идентификаторы, меньше единицы. Рассмотрим случайный процесс $\{ \eta_r \in [0, \overline{N}] \}_{r=1}^\infty$, моделирующий число активных меток в системе. Время в этом процессе дискретно и увлиечивается на единицу после выполнения операции. В дальнейшем, в качестве такой операции будет выступать выполнение раунда и всех действий, определенных спецификацией. Однако, в этом разделе, прежде чем ввести определение операции, моделирующей весь раунд, для удобства будем считать, что $r$ меняется после каждой элементарной операции.

Пусть $U$ "--- одна из элементарных операций, число активных меток перед ее выполнения равно $\eta_r = n$, а после "--- $\eta_{r+1} = n'$, $n,n' \in [0,\overline{N}]$. В силу сделанных ранее допущений, в частности "--- о постоянстве BER во всей области чтения, каждую элементарную операцию будем задавать распределением вероятностей $\mathbb{P}\{\eta_{r+1} = n' | \eta_r = n\}$. Переходные вероятности операции $U$ можно задать в виде матрицы порядка $\overline{N} + 1$. Для удобства будем считать, что нумерация строк и столбцов в матрицах $U$ и элементов вектора $\bm{\pi}$ начинается с нуля, чтобы номера строк и столбцов в точности соответствовали числу активных меток; тогда $\{ U \}_{ij} = \mathbb{P}\{ \eta_{r+1} = j\; |\; \eta_{r} = i \}$. Рассмотрим переходные матрицы каждой из элементарных операций (см. рис.~\ref{fig:ch3_bg_trans}).

\begin{figure}[htb]
	\centerfloat{
    \includegraphics[width=0.9\textwidth]{chapter3/ch3_bg_trans}
  }
  \caption{Изменение числа активных меток при выполнении элементарных операций.}
  \label{fig:ch3_bg_trans}
\end{figure}

Применение операции опроса меток приводит к том, что часть меток передают (возможно, успешно) свои EPCID и инвертируют флаг. Непосредственно после этой операции флаг не меняется, питание не отключается, а метки не покидают и не поступают в область чтения. Поэтому вероятность того, что после опроса меток ровно $n'$ из $n$ будут активны равна $P_n(n - n')$ (см. выражение~\eqref{eq:ch3_pnm}), а матрица $U_N^\nabla$ задается следующим образом:

\begin{equation}\label{eq:ch3_bg_inventory}
	\{ U_N^\nabla \}_{ij} = \begin{cases}
		P_i(i - j), & 0 \leqslant j \leqslant i \leqslant N\\
		1,          & N < i = j \leqslant \overline{N}\\
		0,          & \text{в остальных случаях.}
 	\end{cases}
\end{equation}

В результате инвертирования флага опроса неактивные метки становятся активными, и наоборот, то есть число активных меток после операции равно $N - n$:

\begin{equation}\label{eq:ch3_bg_switch}
	\{ U_N^\times \}_{ij} = \begin{cases}
 		1, & 0 \leqslant i \leqslant N,\; j = N - i\\
 		1, & N < i = j \leqslant \overline{N}\\
 		0, & \text{в остальных случаях.}
 	\end{cases}
\end{equation}

Сброс питания приводит к тому, что у всех меток в области чтения хранимое значение флага становится равным $A$. Так как после этого опрашивать метки по флагу $B$ не имеет практического смысла, будем считать, что флаг опроса также становится равным $A$, то есть все метки "--- активны:

\begin{equation}\label{eq:ch3_bg_power_off}
	\{ U_N^\downarrow \}_{ij} = \begin{cases}
 		1, & 0 \leqslant i \leqslant N,\; j = N\\
 		1, & 0 N < i = j \leqslant \overline{N}\\
 		0, & \text{в остальных случаях.}
 	\end{cases}
\end{equation}

Выполнение операции добавления метки ведет к тому, что в системе становится на одну метку больше (то есть увеличивается $N$). Изменение числа активных меток зависит от текущего значения флага опроса: если флаг опроса $X = A$, то число активных меток увличивается, а если $X = B$, то не изменяется:

\begin{equation}\label{eq:ch3_bg_tag_arrival}
	\begin{aligned}
		\{ U_{A_N}^+ \}_{ij} &= \begin{cases}
			1, & 0 \leqslant i < \overline{N}, \; j = i + 1\\
			1, & N+1 \leqslant i = j \leqslant \overline{N}\\
			0, & \text{в остальных случаях}
	 	\end{cases}\\
		U_{B_N}^+ &= I_{\overline{N}+1},
 	\end{aligned}
\end{equation}
где $I_{\overline{N}+1}$ "--- единичная матрица порядка $\overline{N}+1$.

Наконец, при выходе метки из области чтения число меток $N$ уменьшается. Изменение числа активных меток зависит от того, покинула ли систему активная метка. Здесь необходимо сделать одно замечание. В идеале, нужно учитывать, какая метка находится ближе к области выхода, и смотреть, активна ли она. Однако, это привело бы к серьезному увеличению пространства состояний: вместо одного числа активных меток нужно было бы хранить в состоянии системы положения меток и значения их флагов. Поэтому при определении этой элементарной операции сделаем допущение о том, что покинуть систему может равновероятно любая метка. В этому случае вероятность изменения числа активных меток пропорциональна их числу в системе:

\begin{equation}\label{eq:ch3_bg_tag_departure}
	\{ U_{N}^- \}_{ij} = \begin{cases}
		\frac{i}{N},     & 0 < i \leqslant N,\; j = i - 1\\
		\frac{N - i}{N}, & 0 \leqslant i < N,\; i = j\\
		1,               & N \leqslant i = j \leqslant \overline{N}\\
		0,               & \text{в остальных случаях.}
 	\end{cases}
\end{equation}

Прежде, чем перейти к моделированию раундов инвентаризации, определенных размеченными сценариями, с помощью элементарных операций, сделаем замечание относительно элементов матриц. Можно видеть, что во всех матрицах справа внизу содержится единичный блок. Вообще говоря, строки при $i > N$ не имеют физического смысла: любая операция вида $U^\bullet_N$ применяется к системе, в которой находится $N$ меток. Было бы логично сделать так, чтобы все матрицы имели порядок $N+1$ ($N+2$ для $U_N^+$), то есть чтобы каждый элемент соответствовал возможному изменению числа активных меток, учитывая, что в системе перед выполнением операции находится ровно $N$ меток. Однако, в следующем разделе потребуется рассматривать композиции элементарных операций, для чего их матрицы будут перемножаться. Можно видеть, однако, что вероятность попадания в состояние $n > N$ в результате выполнения любой из операций (за исключением $U_N^+$) равна нулю, и любое из состояний $n > N$ является поглощающим.




%%% --------------------------------------------
\subsection{Построение операций по размеченному сценарию}\label{subsec:ch3_bg_round_op_matrices}
%%% --------------------------------------------
Обозначим распределение вероятностей случайной величины $\eta_r$ как $\bm{\pi}^{(r)} \in \mathbb{R}^{\overline{N}+1}$. Здесь $\pi_n^{(r)} = \mathbb{P}\{\eta_r = n \}$ при $n \in [0, N_r]$, и $\pi_i^{(r)} \equiv 0$ при $i > N_r$. Если перед выполнением элементарной операции $U_i$ распределение вероятностей $\eta$ было $\bm{\pi}$, то распределение после ее выполнения будет $\bm{\pi}U_i$, после выполнения $U_{i+1}$ распределение будет $\bm{\pi} U_i U_{i+1}$ и так далее. Действительно, учитывая, что вероятность изменения $\eta_r$ вследствие выполнения операции определяется только текущим значением $\eta_r$ и типом элементарной операции, случайный процесс $\{ \eta_r \}$ будет неоднородным марковским процессом с дискретным временем, матрицы переходов которого есть матрицы операций.

\begin{figure}[htb]
	\centerfloat{
    \includegraphics[width=1.0\textwidth]{chapter3/ch3_decomposition}
  }
  \caption{Пример представления раундов в виде операций, и декомпозиция этих операций до элементарных операций.}
  \label{fig:ch3_decomposition}
\end{figure}

Представим весь процесс изменения состояния системы в виде последовательности операций $D_r$, по одной операции на раунд опроса. Каждую операцию $D_r$ можно представить в виде композиции элементарных операций (см. пример~\ref{fig:ch3_decomposition}). Отсюда и далее будем полагать, что время в слуайном процессе $\eta_r$ (число активных меток) меняется на единицу в начале каждого раунда. Рассмотрим расширенный сценарий $\widetilde{\bm{\alpha}} = \widetilde{\alpha}_1 \widetilde{\alpha}_2 \dots \widetilde{\alpha}_R$. Пусть $\widetilde{\alpha}_r = X_{N_r}^{e_r}$ и $\widetilde{\alpha}_{r+1} = Y_{N_r}^{e_{r+1}}$. Все, что происходит с системой в $r$-м раунде полностью описывается значениями текущего флага опроса $X$, числом меток $N_r$, признаком сброса питания после опроса $e_r$ и значением флага в следущем раунде $Y$. Чтобы учесть все перечисленные параметры, будем обозначать операцию, моделирующую раунд инвентаризации, как $D_r = D_{X_{N_r},e_r}^{Y_{N_{r+1}}}$. Хотя такое обозначение несколько громоздко, оно позволяет полностью отразить спецификацию раунда. Чтобы немного упростить его, будем опускать $e_r$, когда $e_r = 0$, то есть $D_{X_{N_r},0}^{Y_{N_{r+1}}} \equiv D_{X_{N_r}}^{Y_{N_{r+1}}}$. Также будем опускать число меток в следующем раунде, если $N_{r+1} \equiv N_r$, то есть $D_{X_N,e}^{Y_N} \equiv D_{X_N,e}^Y$.

Используя предложенную нотацию, по размеченному сценарию легко выписать операции. Например, для размеченного сценария $\widetilde{\bm{\alpha}} = A^0_4 B^1_5 A^0_4 A^1_3$ раунды будут описываться операциями $D_{A_4}^{B_5}$, $D_{B_5,1}^{A_4}$, $D_{A_4}^{A_3}$, $D_{A_3,1}^{A_4}$ (напомним, что по определению сценария считыватель начинает его заново после выполнения последнего раунда; распространим эту же договоренность на описание изменений системы с помощью размеченного сценария).

Все возможные операции можно представить в виде композиции элементарных операций следующим образом. Если число меток в системе не меняется, то:
\begin{equation}\label{eq:ch3_bg_ops_1}
	\begin{aligned}
		&D_{A_N}^{A}         = D_{B_N}^{B}         = U_N^\nabla\\
		&D_{A_N}^{B}         = D_{B_N}^{A}         = U_N^\nabla\, U_N^\times\\
		&D_{A_N,1}^{A}       = D_{B_N,1}^{A}       = U_N^\nabla\, U_N^\downarrow\\
		&D_{A_N,1}^{B}       = D_{B_N,1}^{B}       = U_N^\nabla\, U_N^\downarrow\, U_N^\times\\
	\end{aligned}
\end{equation}
Если число меток в системе меняется, но питание после раунда не отключается:
\begin{equation}\label{eq:ch3_bg_ops_2}
	\begin{aligned}
		&D_{A_N}^{A_{N-1}}   = D_{B_N}^{B_{N-1}}                        =      U_N^\nabla\, U_N^-\\
		&D_{A_N}^{B_{N-1}}   = D_{B_N}^{A_{N-1}}                        =      U_N^\nabla\, U_N^-\, U_{N-1}^\times\\
		&D_{A_N}^{A_{N+1}}   = U_N^\nabla\, U_{A_N}^+\\
		&D_{A_N}^{B_{N+1}}   = U_N^\nabla\, U_{A_N}^+\, U_{N+1}^\times\\
		&D_{B_N}^{B_{N+1}}   = U_N^\nabla\, U_{B_N}^+                   \equiv U_N^\nabla\\
		&D_{B_N}^{A_{N+1}}   = U_N^\nabla\, U_{B_N}^+\, U_{N+1}^\times  \equiv U_N^\nabla\, U_{N+1}^\times\\
	\end{aligned}
\end{equation}
Если меняется число меток и отключается питание:
\begin{equation}\label{eq:ch3_bg_ops_3}
	\begin{aligned}
		&D_{A_N,1}^{A_{N-1}} = D_{B_N,1}^{A_{N-1}} = U_N^\nabla\, U_N^\downarrow\, U_N^-\\
		&D_{A_N,1}^{B_{N-1}} = D_{B_N,1}^{B_{N-1}} = U_N^\nabla\, U_N^\downarrow\, U_N^-\, U_{N-1}^\times\\
		&D_{A_N,1}^{A_{N+1}} = D_{B_N,1}^{A_{N+1}} = U_N^\nabla\, U_N^\downarrow\, U_{A_N}^+\\
		&D_{A_N,1}^{B_{N+1}} = D_{B_N,1}^{B_{N+1}} = U_N^\nabla\, U_N^\downarrow\, U_{A_N}^+ \, U_{N+1}^\times
	\end{aligned}
\end{equation}

Вернемся к примеру расширенного сценария $\widetilde{\bm{\alpha}} = A^0_4 B^1_5 A^0_4 A^1_3$. Для него представление в виде операций, моделирующих раунды, и декомпозиция до уровня элементарных операций, будет выглядеть так:

$$
\begin{aligned}
  &\widetilde{\bm{\alpha}} = A^0_4 B^1_5 A^0_4 A^1_3 \\
  &\qquad \Longrightarrow D_{A_4}^{B_5}\, D_{B_5,1}^{A_4}\, D_{A_4}^{A_3}\, D_{A_3,1}^{A_4}\\
  &\qquad \Longrightarrow (U_4^\nabla U_{A_4}^+ U_5^\times) \; (U_5^\nabla U_5^\downarrow U_5^-) \; (U_4^\nabla U_4^-) \; (U_3^\nabla U_3^\downarrow U_{A_3}^+)
\end{aligned}
$$




%%% --------------------------------------------
\subsection{Расчет распределения числа активных меток}
%%% --------------------------------------------
Имея последовательность операций для размеченного сценария, можно рассчитать распределение вероятностей числа активных меток перед началом каждого раунда. Как отмечалось выше, случайный процесс "--- неоднородный марковский. Время в этом процесе увличивается на единицу на каждом раунде, а матрицами переходов являются $D_1, D_2, \dots D_R$. Так как после выполнения сценария сиситема моделируется опять с начала  размеченного сценария, то после операции $D_R$ будут опять применяться операции $D_1, D_2, \dots D_R$. В общем случае можно записать это так:
$$
	D_r \equiv D_{(r - 1)\, (\textbf{mod } R) + 1}.
$$

Если процесс $\eta_r$ на $r$-м шаге имеет распределение $\bm{\pi}^{(r)}$, то после $r$-го раунда распределение будет $\bm{\pi}^{(r)} D_r$, после $r+1$ шага "--- $\bm{\pi}^{(r)} D_r D_{r+1}$ и так далее. Так как после $R$-й переходной матрицы опять используется $D_1$, то распределение процесса на $(r+R)$-м шаге будет
$$
	\bm{\pi}^{(r+R)} = \bm{\pi}^{(r)} D_r D_{r+1} \dots D_R D_1 \dots D_{r-1}.
$$
Обозначим произведение матриц, стоящих в правой части, как $\widetilde{D}_{r}$, то есть положим
$$
	\widetilde{D}_r = D_r D_{r+1} \dots D_R D_1 \dots D_{r-1}.
$$

Последовательность значений процесса $\{ \eta_r \}$ на шагах $r, r+R, r+2R, \dots$ и обозначим эту последовательность как $\{ \eta_i^{[r]} \}_{i=0}^\infty$, $\eta_i^{[r]} = \eta_{r+Ri}$. Так как процесс $\{ \eta_{i}^{[r]} \}$ является подпроцессом марковского процесса $\{ \eta_r \}$, то он и сам является марковским, причем, в отличие от $\{ \eta_r \}$, он будет однородным: переходные матрицы между шагами задаются матрицей $\widetilde{D}_r$. Найти его стационарное распределение $\bm{\pi}^{(r)}$ можно из системы линейных уравнений:
$$
	\begin{cases}
		\bm{\pi}^{(r)} &= \bm{\pi}^{(r)} \widetilde{D}_r\\
		1 &= \sum\limits_{n=0}^{\overline{N}} \pi^{(r)}_n
	\end{cases}
$$
Из найденного вектора $\bm{\pi}^{(r)}$ можно найти $\bm{\pi}^{(r+1)} = \bm{\pi}^{(r)} D_r$, $\bm{\pi}^{(r+2)} = \bm{\pi}^{(r+1)} D_{r+1}$ и так далее. Таким образом, рассчитать распределение вероятностей перед каждым раундом $r = 1, 2, \dots, R$ можно, решив следующую систему уравнений:

\begin{equation}\label{eq:ch3_bg_pmf_system}
	\begin{cases}
		\bm{\pi}^{(1)} &= \bm{\pi}^{(1)} \widetilde{D}_1\\
		\bm{\pi}^{(2)} &= \bm{\pi}^{(1)} D_1\\
		\bm{\pi}^{(3)} &= \bm{\pi}^{(2)} D_2\\
		\dots&\\
		\bm{\pi}^{(R)} &= \bm{\pi}^{(R-1)} D_{R-1}\\
		1              &= \sum\limits_{n=0}^{\overline{N}} \pi^{(1)}_n
	\end{cases}
\end{equation}

Используя найденные распределения числа активных меток можно вычислить новые оценки длительностей раундов $\tau_r = \mathbb{E} \tau_r(n)$, используя формулы~\eqref{eq:ch3_round_duration_of_n} и~\eqref{eq:ch3_round_duration_avg}.



%%% --------------------------------------------
\subsection{Итерационный расчет оценок длительностей раундов}\label{subsec:ch3_iterative_algorithm}
%%% --------------------------------------------
В результате расчета $\tau_r$ по найденным распределениям $\{ \bm{\pi}^{(r)} \}_{r=1}^R$, оценки длительностей раундов могут измениться. Если изначально раунды предполагались сликом короткими или длинными, то после того, как будет найдено распределение числа активных меток, их длительности могут соответственно увеличиться или уменьшиться. Однако, так как моменты поступления и выхода меток из области чтения $\{ t_i^{a} \}_{i=1}^\infty$ и, соответственно, закон $f_N(t)$, не зависят от длительностей раундов, то может измениться и размеченный сценарий.

Рассмотрим пример. Пусть сценарий $\bm{\alpha} = A^0 A^0 A^0 \dots$, в начальный момент в системе две метки и новая метка поступает на 120-й миллисекунде ($t_1^{[a]} = 120$). Предположим изначально, что первые два раунда имеют длительность 100~миллисекунд (то есть $\tau_1 = \tau_2 = 100$). Тогда размеченный сценарий будет иметь вид $\widetilde{\alpha} = A_2^0 A_2^0 A_3^0 \dots$. Если после пересчета выяснится, что раунды имеют длительности $\tau_1 = 150$ и $\tau_2 = 80$, то новый размеченный сценарий будет иметь вид $\widetilde{\alpha}' = A_2^0 A_3^0 \dots$. Соответственно, изменятся и матрицы переходных вероятностей операций $D_r$, поэтому в итоге могут измениться и оценки $\{ \bm{\pi}^{(r)} \}$, и $\{ \tau_r \}$. Таким образом, может потребоваться многократный пересчет оценок $\tau_r$.

Итерационный алгоритм, позволяющий найти оценки длительностей раундов $\{ \tau_r \}$, принимает на вход сценарий $\bm{\alpha} = \alpha_1 \alpha_2 \dots \alpha_R$, закон изменения числа меток $f_N(t)$, максимально допустимую ошибку $\epsilon > 0$ и максимальное число итераций $\overline{K}$. Алгоритм выполняет следующие действия (см. рис.~\ref{fig:ch3_iterative_algorithm}).

\begin{figure}[htb]
	\centerfloat{
    \includegraphics[width=0.7\textwidth]{chapter3/ch3_iterative_algorithm}
  }
  \caption{Итерационный алгоритм расчета длительностей раундов.}
  \label{fig:ch3_iterative_algorithm}
\end{figure}

\textit{Шаг 1}. Вычисляются начальные оценки длительностей раундов $\tau_1, \tau_2, \dots, \tau_R$. При выборе способа нахождения можно использовать разные стратегии. Например, можно предположить, что все раунды имеют минимальную длительность, максимально возможную длительность $\tau_{max}$ (см.~\eqref{eq:ch3_max_round_duration}), или случайную длительность между минимальной и максимальной. Также счетчик числа итераций $k$ выставляется равным единице.

\textit{Шаг 2}. Вычисляется число меток в системе в начале каждого раунда $N_1, N_2, \dots, N_R$, $N_r = f_N(\tau_1 + \dots + \tau_{r-1})$, $\alpha_r = X^{e_r}$.

\textit{Шаг 3}. По найденным значениям $\{ N_r \}$ и заданному сценарию $\bm{\alpha} = \alpha_1 \alpha_2 \dots \alpha_R$ строится размеченный сценарий $\widetilde{\bm{\alpha}} = \widetilde{\alpha}_1 \widetilde{\alpha}_2 \dots \widetilde{\alpha}_R$, $\widetilde{\alpha}_r = X_{N_r}^{e_r}$.

\textit{Шаг 4}. По построенному размеченному сценарию строится последовательность матриц переходных вероятностей $D_1, D_2, \dots, D_R$ процесса $\{ \eta_r \}_{r=1}^\infty$. Каждая матрица $D_r$ строится по соседним символами размеченного сценария $\widetilde{\alpha}_r = X_{N_r}^{e_r}$ и $\widetilde{\alpha}_{r+1} = Y_{N_r}^{e_{r+1}}$ и записывается как $D_{X_{N_r},e_r}^{Y_{N_{r+1}}}$. Для нахождения матрицы используется ее декомпозиция в произведение матриц элементарных операций с помощью выражений~\eqref{eq:ch3_bg_ops_1}"---\eqref{eq:ch3_bg_ops_3}. Матрицы элементарных операций задаются выражениями \eqref{eq:ch3_bg_inventory},~\eqref{eq:ch3_bg_switch},~\eqref{eq:ch3_bg_power_off},~\eqref{eq:ch3_bg_tag_arrival} и \eqref{eq:ch3_bg_tag_departure}.

\textit{Шаг 5}. Вычисляется стационарное распределение числа активных меток перед каждым раундом $\bm{\pi}^{(1)}, \bm{\pi}^{(2)}, \dots, \bm{\pi}^{(R)}$ как решение системы линейных алгебраических уравнений \eqref{eq:ch3_bg_pmf_system}.

\textit{Шаг 6}. С помощью найденных распределений вероятностей $\{ \bm{\pi}^{(r)} \}_{r=1}^R$ вычисляются новые оценки длительностей раундов $\tau'_1, \tau'_2, \dots, \tau'_R$, $\tau'_r = \sum_{n=0}^{\overline{N}} \pi^{(r)}_n \tau(n)$, где длительность раунда для заданного числа активных меток $\tau(n)$ вычисляется с помощью \eqref{eq:ch3_round_duration_of_n}.

\textit{Шаг 7}. Вычисляется среднеквадратичное отклонение между предыдущей и новой оценкой длительностей раундов $\sigma = \sqrt{\sum_{r=1}^R (\tau'_r - \tau_r)^2 / R}$. Если $\sigma \leqslant \epsilon$ или число итераций $k$ достигло максимального $\overline{K}$, то последние вычисленные оценки $\{ \tau'_r \}_{r=1}^R$ возвращаются в качестве результата выполнения алгоритма, и он завершается. В противном случае, если $\sigma > \epsilon$ и $k < \overline{K}$, счетчик числа итераций $k$ увеличивается на единицу, найденные оценки $\{ \tau'_r \}$ используются вместо исходных $\{ \tau_r \}$, и алгоритм переходит на шаг 2.

Следует сделать несколько замечаний относительно работы алгоритма. Во-первых, в общем случае алгоритм может не сойтись к определенному набору значений $\{ \tau_r \}_{r=1}^R$ со сколь угодно малой точностью. Это связано с тем, что при нахождении оценок могут возникать эффекты, сходные с колебаниями: при одной оценке события изменения числа меток происходят в одних раундах, а при пересчете оценок "--- в других; но после еще одной или нескольких переоценки длительностей они опять начинают происходить в тех же раундах. Для того, чтобы алгоритм наверняка завершался приходится вводит ограничение на число итераций. В то же время, алгоритм в любом случае позволяет отойти от слишком грубых оценок длительностей раундов.

Во-вторых, в качестве входа стоит использовать не сам сценарий $\bm{\alpha} = \alpha_1 \alpha_2 \dots \alpha_R$, а его многократно увеличинную версию вида
$$
	\bm{\alpha}^{[L]} = \underbrace{\bm{\alpha} \dots\, \bm{\alpha}}_{L \text{ повторов}} =
		\underbrace{\alpha_1 \dots\, \alpha_R \; \alpha_1 \dots\, \alpha_R\; \dots \, \alpha_1 \dots\, \alpha_R}_{LR \text{ символов}}.
$$
Это приходится делать для того, чтобы сгладить <<краевые эффекты>>, возникающие при зацикливании размеченных раундов: при определении матрицы переходных вероятностей $D_R = D_{X_{N_{R}},e_R}^{Y_{N_1}}$ используются спецификации $R$-го и 1-раундов. Однако, возможна ситуация, когда число меток изменилось непосредственно перед последним раундом. Например, обычно матки появляются или исчезают в среднем каждые 20 раундов, а последнее изменение произошлов на $(R-1)$-м раунде. В этом случае переход после $R$-го раунда будет выбиваться и может привести к появлению погрешности. Поэтому желательно иметь сценарий такой длинны, чтобы в нем поместилось достаточно большое число событий добавления и удаления меток из области чтения. В качестве метрики для определения чила повторов $L$ можно использовать отношение $LR \approx J \lfloor \underline{\Delta t} / \tau_{max} \rfloor$, где коэффициент $J \geqslant 10$, причем чем больше, тем лучше. Нужно, однако, учесть, что увеличение длины сценария ведет к увеличению времени расчета.

В-третьих, желательно начинать моделирование с того момента, когда в системе уже находится некоторое количество меток. Так как переходная матрица $D_R$ строится по символам расширенного сценария $\widetilde{\alpha}_R = X_{N_R}^{e_r}$ и $\widetilde{\alpha}_1 = Y_{N_1}^{e_1}$, то если начинать моделирования с момента $t = 0$, для которого $f_N(t) = 0$, а в среднем в системе находится $N_a = \lim_{T \rightarrow \infty} \frac{1}{T} \int_{t=0}^{T} f_N(t) dt > 1$ меток, то в результате перехода вся информация о том, в каких состояниях находились метки, потеряется, и распределение вероятностей будет $\bm{\pi}^{(R+1)} \equiv \bm{\pi}^{(0)} \equiv (1, 0, \dots, 0)$. При этом все метки после $R$-го раунда как будто внезапно исчезнут из системы. Чтобы этого избежать, желательно начинать моделирование с некоторго момента $t_0$, в который $f_N(t_0) \approx N_a$. Например, если метки поступают через равные промежутки времени $\Delta t$, можно выбрать такой момент, когда в области чтения находится $\lfloor T_L / \Delta t \rfloor$ меток.

После того, как оценки длительностей раундов $\{ \tau_r \}$ и число меток в системе в начале каждого раунда $\{ N_r \}$ найдены, можно рассчитать вероятность идентификации отдельной метки.






%%%%%%%%%%%%%%%%%%%%%%%%%%%%%%%%%%%%%%%%%%%%%%%%%%%%%%%%%%%%%%%%%%%%%%%%%%%%%%%%
\section{Расчет вероятности идентификации}
%%%%%%%%%%%%%%%%%%%%%%%%%%%%%%%%%%%%%%%%%%%%%%%%%%%%%%%%%%%%%%%%%%%%%%%%%%%%%%%%
После выполнения итерационного алгоритма, известны оценки длительностей раундов $\{ \tau_r \}_{r=1}^R$, число меток в системе в начале каждого раундв $\{ N_r \}_{r=1}^R$, размеченный сценарий $\widetilde{\bm{\alpha}} = \widetilde{\alpha}_1 \widetilde{\alpha}_2 \dots \widetilde{\alpha}_R$, $\alpha_r = X_{N_r}^{e_r}$ и распределения числа активных меток в каждом раунде $\{ \bm{\pi}^{(r)} \}_{r=1}^R$.

Для расчета вероятности идентификации рассмотрим проезд области чтения одной выделенной меткой. Система при этом ведет себя как обычно: считыватель работает, а метки поступают и покидают систему по сценарию $\bm{\widetilde{\alpha}}$. Выберем такой раунд $r_0$, что $N_{r_0-1} < N_{r_0}$, то есть перед раундом $r_0$ в систему поступила новая метка. Время нахождения метки в области известно и равно $T_L$ по условию задачи. Определим число $Q_{r_0}$ как количество раундов, в течение которых метка, поступившая в раунде $r_0$ находится в системе, то есть:

$$
\tau_{r_0} + \tau_{r_0 + 1} + \dots + \tau_{r_0 + Q_{r_0} - 2} < T_L \leqslant \tau_{r_0} + \tau_{r_0 + 1} + \dots + \tau_{r_0 + Q_{r_0} - 1},
$$
где $\tau_{r} \equiv \tau_{(r - 1)(\textbf{mod } R) + 1}$, то есть $\tau_{R+1} \equiv \tau_1$, $\tau_{R+2} \equiv \tau_2$ и так далее. Отметим, что согласно сделанным ранее допущениям, метка, находящаяся в системе в начале раунда, точно в нем участвует, поэтому время в $Q_{r_0}$ раундах, начиная с $r_0$-го, должно быть не меньше, чем $T_L$ (правая часть равенства).

Рассмотрим конечный двумерный случайный процесс $\{ (\eta_r^{[r_0]}, \phi_r^{[r_0]}) \}_{r=1}^{Q_{r_0}}$, в котором $\eta_r^{[r_0]} \equiv \eta_{r+r_0-1}$ "--- число активных меток в $(r + r_0 - 1)$-м раунде, а $\phi_r$ "--- состояние выделенной метки: если метка еще ни разу не передала успешно свой идентификатор (EPCID или комбинацию EPCID и TID), то $\phi_r^{[r_0]} = 0$, если метка неактивна, и $\phi_r^{[r_0]} = 1$, если активна. Если же выделенная метка хотя бы раз успешно передала свой идентификатор, то $\phi_r^{[r_0]} = 2$ независимо от того, активна ли она, то есть $\phi_r^{[r_0]} = 2$ "--- поглощающее состояние процесса $\{ \phi_r \}$. Вероятность идентификации наблюдаемой метки при условии, что она поступила в раунде $r_0$, можно выразить как $\mathbb{P}\{ \phi_{Q_{r_0}}^{[r_0]} = 2\}$.

Номер раунда $r_0$, в котором поступила наблюдаемая метка, определяется не однозначно. Разумно предположить, что вероятность выбора такого раунда пропорциональна его длительности. Положим $N_0 = N_R$ и обозначим множество всевозможных номеров раундов, в которых происходило увеличение числа меток в системе, как $\mathfrak{R}$:

$$
	\mathfrak{R} = \{ r\:|\:r \in [1, R],\; N_{r-1} < N_r \},
$$
а вероятность поступления метки в раунде $r_0 \in \mathfrak{R}$ оценим как:
\begin{equation}\label{eq:ch3_fg_prob_arrival}
	p^{[a]}_{r_0} = \frac{\tau_{r_0}}{\sum_{r \in \mathfrak{R}} \tau_r}
\end{equation}
Используя введенные обозначения, вероятность идентификации метки можно выразить с помощью полной вероятности:
\begin{equation}\label{eq:ch3_tag_id_prob_phi}
	P_X = \sum\limits_{r \in \mathfrak{R}} p^{[a]}_r \mathbb{P}\{ \phi^{[r]}_{Q_r} = 2 \}.
\end{equation}

Таким образом, для нахождения вероятности идентификации метки нужно найти вероятности идентификации метки $\mathbb{P}\{ \phi^{[r_0]}_{Q_{r_0}} = 2 \}$ при условии поступления ее в раунде $r_0 \in \mathfrak{R}$. Стоит отметить, что процесс $\{ \phi^{[r_0]}_r \}$ не является марковским, так как вероятность его перехода между состояниями 0 и 1 зависит не только от его текущего значения, но и от того, сколько в системе активных меток. Например, если $\phi_r^{[r_0]} = 0$ (наблюдаемая метка неактивна), и других активных меток в системе нет, то вероятность перехода $\phi_r^{[r_0]}$ в состояние 1 или 2 определяется только вероятностью успешной передачи ответа RN16 меткой в очередном раунде; если же в системе есть активные метки, то нужно также учитывать вероятность коллизий.



%%% --------------------------------------------
% \subsection{Определение основного процесса \texorpdfstring{\( \{ \gamma_r \} \)}{γ\_\{r\}}}
\subsection{Определение основного процесса}
%%% --------------------------------------------
% \subsection{Заголовки с формулами: \texorpdfstring{\(a^2 + b^2 = c^2\)}{%
% a\texttwosuperior\ + b\texttwosuperior\ = c\texttwosuperior},
% \texorpdfstring{\(\left\vert\textrm{{Im}}\Sigma\left(
% \protect\varepsilon\right)\right\vert\approx const\)}{|ImΣ (ε)| ≈ const},
% \texorpdfstring{\(\sigma_{xx}^{(1)}\)}{σ\_\{xx\}\textasciicircum\{(1)\}}
% }\label{subsec:with_math}

Рассмотрим двумерный процесс $\{ (\eta_r^{[r_0]}, \phi_r^{[r_0]}) \}_{r=1}^{Q_{r_0}}$. Во всех дальнейших расчетах вплоть до вычисления вероятности $P_X$ будем использовать процесс, начинающийся в раунде $r_0$, поэтому, для сокращения записи, будем опускать верхний индекс $[r_0]$, но подразумевать его. Таким образом, вместо $\phi_r^{[r_0]}$ будем писать $\phi_r$, вместо $Q_{r_0}$ "--- просто $Q$ и так далее.

Изменения компонентов процесса $\{ (\eta_r, \phi_r) \}$ не являются независимыми. Например, если $\eta_r = 0$, вероятность того, что $\phi_r = 1$ равна нулю: наблюдаемая метка не может быть активной, если активных меток в системе нет. Кроме того, первая компонента процесса ($\eta_r$) нужна только для того, чтобы определить вероятности переходов, для расчета $P_X$ нужно лишь понять, оказался ли процесс $\phi_r$ в поглощающем состоянии за $Q$ шагов. Чтобы избавиться от недостижимых состояний и не рассматривать $\eta_r$ после того, как $\phi_r$ оказался в поглощающем состоянии, вместо двумерного процесса будем рассматривать одномерный процесс $\{ \gamma_r \}_{r=1}^Q$, $\gamma_r \in [1, 2\overline{N}+1]$, в котором:

\begin{equation}\label{eq:ch3_gamma_process}
	\begin{aligned}
		1 \leqslant \gamma_r \leqslant \overline{N}                 &\Leftrightarrow \phi_r = 1 \text{ и } \eta_r = \gamma_r \\
		\overline{N} + 1 \leqslant \gamma_r \leqslant 2\overline{N} &\Leftrightarrow \phi_r = 0 \text{ и } \eta_r = \gamma_r - \overline{N} - 1\\
		\gamma_r = 2\overline{N}+1                                  &\Leftrightarrow \phi_r = 2
	\end{aligned}
\end{equation}

Попадание в состояние $\gamma_r = 2\overline{N} + 1$ означает, что наблюдаемая метка успешно передала свой идентификатор, это состояние является поглощающим. При $\gamma_r \leqslant \overline{N}$ число активных меток $\eta_r$ равно $\gamma_r$, причем наблюдаемая метка активна. Если же процесс $\gamma_r$ находится в состоянии из интервала $[\overline{N}+1, 2\overline{N}]$, то наблюдаемая метка неактивна, и число активных меток $\eta_r$ определяется как $\gamma_r - \Delta N$, где $\Delta N = \overline{N} + 1$. Легко видеть, что для состояний $\eta_r = 0, \phi_r = 1$ (активных меток в системе нет, но наблюдаемая метка активна), а также $\eta_r = \overline{N}, \phi_r = 0$ (все метки активны, но наблюдаемая метка неактивна), нет соответствующих состояний $\gamma_r$; кроме того, значения $\eta_r$ неразличимы при $\gamma_r = 2\overline{N} + 1$.

Переходные вероятности для процесса $\{ \gamma_r \}$ найдем аналогично тому, как ранее определялись переходные вероятности для случайного процесса $\{ \eta_r \}_{r=1}^\infty$: сначала покажем, что вероятности переходов $\{ \gamma_r \}$ при выполнении элементарных операций зависят только от текущего состояния и выпишем их, а затем покажем, как построить матрицы переходных вероятностей между раундами.



%%% --------------------------------------------
% \subsection{Переходные вероятности процесса \texorpdfstring{\( \{ \gamma_r \} \)}{γ\_\{r\}} для элементарных операций}
\subsection{Переходные вероятности основного процесса для элементарных операций}
%%% --------------------------------------------
Для обозначений переходных вероятностей процесса $\{ \gamma_r \}$ при выполнении элементарных операций будем использовать нотацию, аналогичную обозначениям для процесса $\{ \eta_r \}$ ($U_N^{\nabla}, U_N^{\times}, \dots$), но вместо буквы $U$ использовать $V$. Как и в разделе \ref{subsec:ch3_bg_elem_op_matrices}, здесь для удобства будем предполагать, что состояние процесса $\{ \gamma_r \}$ изменяется при выполнении одной из элементарных операций: инвентаризации ($V_N^\nabla$), смены флага ($V_N^\times$), сброса питания ($V_N^\downarrow$), добавления или удаления метки ($V^+_{X_N}$ и $V^-_N$ соответственно). В дальнейшем же шаг процесса будет, как обычно, соответствовать раунду.

Пусть $V$ "--- одна из элементарных операций, состояние процесса $\gamma_r$ перед ее выполнением есть $\gamma_r = i$, а после "--- $\gamma_{r+1} = j$, $i,j \in [1,\;2\overline{N}+1]$. Как и ранее для процесса $\{ \eta_r \}$, в силу сделанных допущений, в частности "--- о постоянстве BER во всей области чтения, каждую элементарную операцию можно задать распределением вероятностей $\mathbb{P}\{\gamma_{r+1} = j | \gamma_r = i\}$. Эти вероятности определяются аналогично тому, как это было сделано для переходных вероятностей $\{ \eta_r \}$, однако здесь приходится отдельно рассматривать переходы при $\phi_r = 0$ и $\phi_r = 1$. Рассмотрим переходные матрицы $V \in \mathbb{R}^{(2\overline{N}+1) \times (2\overline{N}+1)}$ каждой из элементарных операций (см. рис.~\ref{fig:ch3_bg_trans}). Как и в разделе $\ref{subsec:ch3_bg_elem_op_matrices}$, некорркетные состояния $j \in [N+1, \overline{N}] \cup [\overline{N} + N + 1,\; 2\overline{N}]$, которые соответствуют $\eta_r > N$ (то есть в которых активных меток больше, чем всего меток в системе), не будут достижимы из корректных состояний $i \in [1,\;N] \cup [\overline{N}+1,\;\overline{N} + N]$, а у них самих будут только переходы-петли с вероятностью 1. Единственное исключение из этого правила "--- операция добавления метки $V_{A_N}^+$.


Применение операции опроса меток $V_N^\nabla$ приводит к тому, что часть активных меток успешно передают RN16 и затем пытаются передать EPCID. Если одной из таких меток оказалась наблюдаемая метка (то есть $n \leqslant \overline{N}$), то процесс может перейти в поглощающее состояние, если наблюдаемая метка сможет полностью передать свой идентификатор. Если же ей это не удастся, то она может либо остаться активной (случай неуспешной передачи RN16), или же инвертировать флаг, но ошибиться уже после передачи RN16. При построении вероятностей удобно рассматривать три независимых события: $(i - j)$ из $i$ активных меток успешно передали свои RN16 (вероятность $P_i(i - j)$); наблюдаемая метка попала в число меток, успешно передавших RN16 (вероятность $(i - j)/i$); наблюдаемая метка успешно передала свой идентификатор (вероятность $P_{\text{ID}}$). С учетом этого замечания и определения состояний процесса $\{ \gamma_r \}$ (см.~\eqref{eq:ch3_gamma_process}), вероятности определяются следующим образом:

\begin{equation}\label{eq:ch3_fg_inventory}
	\{ V^\nabla_N \}_{ij} = \begin{cases}
		\frac{j}{i}P_i(i - j), &1 \leqslant j \leqslant i \leqslant N\\
		1, & N < i = j \leqslant \overline{N}\\
		\mathrlap{\frac{i - (j - \Delta N)}{i} (1 - P_{\text{ID}}) P_i(i - (j - \Delta N)),}\\
			\mathrlap{\qquad\qquad\qquad 1 \leqslant i \leqslant N, \; \overline{N} < j \leqslant \overline{N} + N}\\
		P_{i - \Delta N}(i - j), &\overline{N} < j \leqslant i \leqslant \overline{N} + N\\
		1, & \overline{N} + N < i = j \leqslant 2\overline{N}\\
		P_{\text{ID}} \sum_{k = 1}^i \frac{k}{i} P_i(k), & 1 \leqslant i \leqslant N,\; j = 2\overline{N}+1\\
		0, &\text{в остальных случаях.}
 	\end{cases}
\end{equation}

При смене флага активные и неактивные метки меняются местами. Это справедливо и для наблюдаемой метки, поэтому значение $\phi_r$ меняется с 1 и 0 и обратно. Учитывая определение \eqref{eq:ch3_gamma_process}, получаем следующие значения переходных вероятностей:

\begin{equation}\label{eq:ch3_fg_switch}
	\{ V_N^\times \}_{ij} = \begin{cases}
 		1, & 1 \leqslant i \leqslant N, \; j = \Delta N + (N - i)\\
 		1, & \overline{N} < i \leqslant \overline{N} + N,\; j = N - (i - \Delta N)\\
 		1, & N < i = j \leqslant \overline{N}\\
 		1, & \overline{N} + N < i = j \leqslant 2\overline{N} + 1\\
 		0, & \text{в остальных случаях.}
 	\end{cases}
\end{equation}

В результате выключения питания все метки сбрасывают свои флаги в $A$, а считыватель переключается на этот же флаг. Поэтому из любого корректного непоглощающего состояния $i \in [1, N] \cup [\overline{N}+1, \overline{N} + N]$ она переводит процесс в состояние $\gamma_{r+1} = N$:

\begin{equation}\label{eq:ch3_fg_power_off}
	\{ V_N^\downarrow \}_{ij} = \begin{cases}
 		1, & 1 \leqslant i \leqslant N,\; j = N\\
 		1, & \overline{N} < i \leqslant \overline{N} + N,\; j = N\\
 		1, & N < i = j \leqslant \overline{N}\\
 		1, & \overline{N} + N < i = j \leqslant 2\overline{N} + 1\\
 		0, & \text{в остальных случаях}.
 	\end{cases}
\end{equation}

Операция добавления метки никак не влияет на состояние наблюдаемой метки $\phi_r$, и изменяет число активных меток только в том случае, если текущий флаг опроса равен $A$:

\begin{equation}\label{eq:ch3_fg_tag_arrival}
	\begin{aligned}
		\{ V_{A_N}^+ \}_{ij} &= \begin{cases}
 			1, & 1 \leqslant i < \overline{N},\; j = i + 1\\
 			1, & \overline{N} < i \leqslant \overline{N} + N  < 2\overline{N},\; j = i + 1\\
	 		1, & N < i = j \leqslant \overline{N}\\
 			1, & \overline{N} + N < i = j \leqslant 2\overline{N} + 1\\
 			0, & \text{в остальных случаях}
	 	\end{cases}\\
	 	V_{B_N}^+ &= I_{2\overline{N}+1}
	\end{aligned}
\end{equation}
Здесь $I_{2\overline{N}+1}$ "--- единичная матрица порядка $2\overline{N}+1$.

При моделировании выхода метки из области чтения будем предполагать, что выйти может любая из меток, кроме наблюдаемой. Отметим, что из-за этого допущения возможна некоторая неточность, так как в действительности метки выходят в соответствии с их законом движения, и вероятность выхода метки с определенным флагом $X$ может не быть равна отношению числа меток с флагом $X$ к общему числу меток; но это допущение существенно упрощает модель и позволяет не учитывать положения меток. Про наблюдаемую же метку заведомо известно, что она остается в системе, поэтому при расчете вероятностей ее учитывать не нужно. В остальном, вероятности определяются так же, как и в~\eqref{eq:ch3_bg_tag_departure}, с поправкой на структуру состояний $\{ \gamma_r \}$.

\begin{equation}\label{eq:ch3_fg_tag_departure}
	\{ V_N^- \}_{ij} = \begin{cases}
 		\frac{i-1}{N-1}, & 1 < i \leqslant N,\; j = i - 1\\
 		\frac{N-i}{N-1}, & 1 \leqslant i = j \leqslant N\\
 		1,               & N < i = j \leqslant \overline{N}\\
 		\frac{i - \Delta N}{N - 1},            & \overline{N} + 1 < i \leqslant \overline{N} + N,\; j = i - 1\\
 		\frac{(N-1) - (i - \Delta N)}{N - 1},  & \overline{N} + 1 \leqslant i = j \leqslant \overline{N} + N\\
 		1,               & \overline{N} + N < i = j \leqslant 2\overline{N}\\
 		1,               & i = j = 2\overline{N}+1\\
 		0,               & \text{в остальных случаях}.
 	\end{cases}
\end{equation}





%%% --------------------------------------------
% \subsection{Матрицы переходных вероятностей между раундами для процесса \texorpdfstring{\( \{ \gamma_r \} \)}{γ\_\{r\}}}
\subsection{Матрицы переходных вероятностей между раундами для основного процесса}
%%% --------------------------------------------
Здесь и далее будем считать, что состояние процесса $\{ \gamma_r \}$ меняется между раундами. Как ранее для процесса $\{ \eta_r \}$, из матриц элементарных операций $V_N^\nabla, V_N^\times, V_N^\downarrow, V_{X_N}^+, V_N^-$ можно получить переходные вероятности $C_r^{[r_0]}$ в виде произведения матриц элементарных операций (см. рис. ~\ref{fig:ch3_decomposition}). Если в размеченном сценарии $\widetilde{\bm{\alpha}}$ спецификация $(r_0 + r)$-го раунда есть $\widetilde{\alpha}_{r_0 + r} = X_{N}^e$, а $(r_0+r+1)$-го раунда "--- $\widetilde{\alpha}_{r_0+r+1} = Y_{N'}^{e'}$, то матрицу переходных вероятностей будем обозначать как $C_r^{[r_0]} = C_{X_N,e}^{Y_{N'}}$. Так же, как и для матриц $D_{X_{N},e}^{Y_{N'}}$, будем опускать число меток $N'$ в верхнем индексе, если $N' = N$, и не будем указывать в нижнем индексе $e$, если $e = 0$. В тех случаях, когда структура матрицы не представляет интереса (говоря о переходе после $r$-го раунда) и индекс первого раунда $r_0$ не играет роли, будем опускать верхний индекс $[r_0]$ у матриц и писать $C_r = C_r^{[r_0]}$.

Матрицы $C_{X_N,e}^{Y_{N'}}$ строятся точно так же, как и матрицы $D_{X_N,e}^{Y_{N'}}$, с поправкой на то, что вместо матриц элементарных операций $U_N^\nabla$, $U_N^\times$, $U_N^\downarrow$, $U_{X_N}^+$ и $U_N^-$ используются $V_N^\nabla$, $V_N^\times$, $V_N^\downarrow$, $V_{X_N}^+$ и $V_N^-$.  Для раундов, в которых может сбрасываться питание или меняться флаг, но число меток в системе постоянно:
\begin{equation}\label{eq:ch3_fg_ops_1}
	\begin{aligned}
		&C_{A_N}^{A}         = C_{B_N}^{B}         = V_N^\nabla\\
		&C_{A_N}^{B}         = C_{B_N}^{A}         = V_N^\nabla\, V_N^\times\\
		&C_{A_N,1}^{A}       = C_{B_N,1}^{A}       = V_N^\nabla\, V_N^\downarrow\\
		&C_{A_N,1}^{B}       = C_{B_N,1}^{B}       = V_N^\nabla\, V_N^\downarrow\, V_N^\times\\
	\end{aligned}
\end{equation}
Если число меток в системе меняется, но питание после раунда не отключается:
\begin{equation}\label{eq:ch3_fg_ops_2}
	\begin{aligned}
		&C_{A_N}^{A_{N-1}}   = C_{B_N}^{B_{N-1}}                        =      V_N^\nabla\, V_N^-\\
		&C_{A_N}^{B_{N-1}}   = C_{B_N}^{A_{N-1}}                        =      V_N^\nabla\, V_N^-\, V_{N-1}^\times\\
		&C_{A_N}^{A_{N+1}}   = V_N^\nabla\, V_{A_N}^+\\
		&C_{A_N}^{B_{N+1}}   = V_N^\nabla\, V_{A_N}^+\, V_{N+1}^\times\\
		&C_{B_N}^{B_{N+1}}   = V_N^\nabla\, V_{B_N}^+                   \equiv V_N^\nabla\\
		&C_{B_N}^{A_{N+1}}   = V_N^\nabla\, V_{B_N}^+\, V_{N+1}^\times  \equiv V_N^\nabla\, V_{N+1}^\times\\
	\end{aligned}
\end{equation}
Если меняется число меток и отключается питание:
\begin{equation}\label{eq:ch3_fg_ops_3}
	\begin{aligned}
		&C_{A_N,1}^{A_{N-1}} = C_{B_N,1}^{A_{N-1}} = V_N^\nabla\, V_N^\downarrow\, V_N^-\\
		&C_{A_N,1}^{B_{N-1}} = C_{B_N,1}^{B_{N-1}} = V_N^\nabla\, V_N^\downarrow\, V_N^-\, V_{N-1}^\times\\
		&C_{A_N,1}^{A_{N+1}} = C_{B_N,1}^{A_{N+1}} = V_N^\nabla\, V_N^\downarrow\, V_{A_N}^+\\
		&C_{A_N,1}^{B_{N+1}} = C_{B_N,1}^{B_{N+1}} = V_N^\nabla\, V_N^\downarrow\, V_{A_N}^+ \, V_{N+1}^\times
	\end{aligned}
\end{equation}




%%% --------------------------------------------
% \subsection{Расчет вероятности поглощения процесса \texorpdfstring{\( \{ \gamma_r \} \)}{γ\_\{r\}}}
\subsection{Расчет вероятности поглощения процесса}
%%% --------------------------------------------
Зная переходные вероятности процесса $\{ \gamma_r^{[r_0]} \}$, можно вычислить вероятность того, что метка, поступивашая в систему в раунде $r_0 \in \mathfrak{R}$ успешно передаст свой идентификатор. Обозначим $\bm{\theta}^{(r_0,r)} \in \mathbb{R}^{2\overline{N}+1}$ распределение вероятностей процесса $\gamma_r^{[r_0]}$ на $r$-м шаге, а $\bm{\theta}^{(r_0,1)}$ "--- его начальное распределение. Так как вероятности переходов между $r$-м и $(r+1)$-м состоянием определяются матрицей $C_r^{[r_0]}$, то
$$
\bm{\theta}^{(r_0,r+1)} = \bm{\theta}^{(r_0,r)} C_{r}^{[r_0]} = \bm{\theta}^{(r_0,1)} C_1^{[r_0]} C_2^{[r_0]} \dots C_r^{[r_0]}.
$$
В частности, распределение вероятностей после $Q_{r_0}$ раундов можно вычислить как:
$$
\bm{\theta}^{(r_0, Q_{r_0} + 1)} = \bm{\theta}^{(r_0, 1)} C^{[r_0]}_1 C^{[r_0]}_2 \dots C^{[r_0]}_{Q_{r_0}}.
$$

В силу определения состояний процесса $\{ \gamma_r^{[r_0]} \}$ \eqref{eq:ch3_gamma_process}, вероятность попадания процесса $\{ \phi_r^{[r_0]} \}$ в поглощающее состояние $\phi_r^{[r_0]} = 2$ тождественно равна вероятности попадания процесса $\{ \gamma_r^{[r_0]} \}$ в его поглощающее состояние $\gamma_{Q_{r_0}}^{[r_0]} = 2\overline{N}+1$, то есть
$$
	\mathbb{P}\{ \phi_{Q_{r_0}}^{[r_0]} = 2 \} \equiv \mathbb{P}\{ \gamma_{Q_{r_0}}^{[r_0]} = 2\overline{N}+1 \} = \theta_{2\overline{N}+1}^{(r_0, Q_{r_0}+1)}.
$$

Таким образом, для вычисления вероятности поглощения процесса $\{ \phi_r^{[r_0]} \}$, то есть успешной идентификации наблюдаемой метки, поступившей в систему в раунде $r_0$, нужно вычислить распределение вероятностей после выполнения $Q_{r_0}$ раундов. Так как размеченный сценарий $\widetilde{\bm{\alpha}}$, по которому строятся матрицы переходных вероятностей $C^{[r_0]}_1$, $C^{[r_0]}_2$, $\dots$, $C^{[r_0]}_{Q_{r_0}}$, был построен одновременно с вычислением оценок длительностей раундов $\tau_1, \tau_2, \dots, \tau_R$ в результате выполнения итерационного алгоритма (см. раздел~\ref{subsec:ch3_iterative_algorithm}), то неизвстным остается только начальное распределение $\bm{\theta}^{(r_0, 1)}$.

Для нахождения начального распределения $\bm{\theta}^{(r_0,1)}$ нужно учесть, что об этом распределении уже кое-что известно. Во-первых, известно, будет ли наблюдаемая метка активна сразу после поступления: она будет активной, если и только если в раунде $r_0$ считыватель ведет опрос по флагу $A$, поскольку новая метка всегда хранит флаг со значением $A$. Значит, если $\alpha_{r_0} = A^{e}$, то значение $\phi_1^{[r_0]} = 0$, и, соответственно, $\gamma_1^{[r_0]} \in [1,\; N]$; если же $\alpha_{r_0} = B^{e}$, то $\phi_1^{[r_0]} = 1$ и $\gamma_1^{[r_0]} \in [\overline{N} + 1,\; \overline{N} + N]$ Во-вторых, значение $N$ известно из символа $\widetilde{\alpha}_{r_0} = X^e_N$ размеченного сценария. Наконец, в-третьих, известно распределение числа активных меток $\eta_{r_0}$ в начале раунда $\bm{\pi}^{(r_0)}$ "--- оно было найдено в результате выполнения итерационного алгоритма. Таким образом:
\begin{equation}\label{eq:ch3_fg_initial_prob}
	\theta_n^{(r_0,1)} = \begin{cases}
		\pi^{(r_0)}_n,                      &\widetilde{\alpha}_{r_0} = A^e_N,\; 1 \leqslant n \leqslant N\\
		\pi^{(r_0)}_{n - (\overline{N}+1)}, &\widetilde{\alpha}_{r_0} = B^e_N,\; \overline{N}+1 \leqslant n \leqslant \overline{N}+N\\
		0,                                  &\text{в остальных случаях.}
	\end{cases}
\end{equation}

Подставляя найденные выражения для вероятности попадания $\{ \gamma_r^{[r_0]} \}$ в поглощающее состояние в выражение~\eqref{eq:ch3_tag_id_prob_phi}, получаем, что вероятность идентификации наблюдаемой метки $P_X$ вычисляется так:

$$
	P_X = \sum\limits_{r_0 \in \mathfrak{R}} p_{r_0}^{[a]} \{ \bm{\theta}^{(r_0,1)} C_1^{[r_0]} C_2^{[r_0]} \dots C_{Q_{r_0}}^{[r_0]} \}_{2\overline{N}+1},
$$
где вероятности $p_{r_0}^{[a]}$ вычисляются согласно~\eqref{eq:ch3_fg_tag_arrival}, матрицы переходных вероятностей "--- согласно~\eqref{eq:ch3_fg_ops_1}-\eqref{eq:ch3_fg_ops_3}, а начальные распределения $\bm{\theta}^{(r_0,1)}$ "--- согласно~\eqref{eq:ch3_fg_initial_prob}.






%%%%%%%%%%%%%%%%%%%%%%%%%%%%%%%%%%%%%%%%%%%%%%%%%%%%%%%%%%%%%%%%%%%%%%%%%%%%%%%%
\section{Результаты моделирования}
%%%%%%%%%%%%%%%%%%%%%%%%%%%%%%%%%%%%%%%%%%%%%%%%%%%%%%%%%%%%%%%%%%%%%%%%%%%%%%%%
Для численного исследования был выбран простой случай проезда автомобилями области чтения с постоянной скоростью 5 м/с (то есть 18~км/ч). Предполагалось, что метки размещаются только на передних номерах или лобовых стеклах. Интервал между метками составлял 5 метров (длина автомобиля около 3 метров, интервал между автомобилями "--- 2 метра), то есть метки поступали в моменты $a_i = 0, 1, 2, \dots$~сек.; длина области чтения предполагалась равной $L=13$ метров, так что одновременно в области чтения находилось от двух до трех меток, и каждая метка находилась в области чтения $T_L = 2,6$~с. С учетом сделанных предположений, функция числа меток в системе $f_N(t)$ определялась как:
$$
	f_N(t) = \begin{cases}
		1, &t < 1\\
		2, &1 \leqslant t < 2 \text{ или } k + 0,6 \leqslant t < k+1 \text{ для } k \geqslant 2 \\
		3, &k \leqslant t < k + 0,6 \text{ для } k \geqslant 2
	\end{cases}
$$

Считыватель использовал значения Tari = 6,25~мкс., DR = 64/3, M = 2 и Q = 2, а BER менялся от 0 до 0.2. Питание сбрасывалось на $T_{\downarrow} = 0,1$~сек.. Рассматривалось 5 сценариев (отметим, что, согласно замечанию к итерационному алгортму, при расчетах каждый из сценариев кратно удлиннялся):
\begin{enumerate}
	\item $A^0$ "--- опрос меток по флагу $A$ без сбросов питания;
	\item $A^1$ "--- опрос по флагу $A$, сброс питания после каждого раунда;
	\item $\underbrace{A^0 \dots\, A^0}_{7 символов} A^1$: опрос по $A$, сброс питания каждые 8 раундов;
	\item $A^0 A^0 B^0 B^0$: смена флага каждые два раунда без сброса питания;
	\item $A^0 A^0 A^0 A^0 B^0 B^0 B^0 B^0$: смена флага каждые четрые раунда.
\end{enumerate}

\begin{figure}[htb]
	\centerfloat{
    \includegraphics[width=0.75\textwidth]{chapter3/ch3_plot_base}
  }
  \caption{Вероятность идентификации метки при опросе по флагу $A$ без сброса питания, идентификация по EPCID или по комбинации EPCID и TID.}
  \label{fig:ch3_plot_base}
\end{figure}

Для валидации полученных результатов использовалась имитационная модель. В первом случае (см. рис.~\ref{fig:ch3_plot_base}) вероятность быстро падала с ростом BER, так как любая метка, успешно передавшая RN16, но не сумевшая передать EPCID без ошибки, инвертировала свой флаг и больше не принимала участие в опросах. Ожидаемо, вероятность при идентификации только по EPCID несколько лучше, чем при идентификации по паре EPCID и TID, но все равно мала. Этот сценарий рассматривался далее как худший вариант.

\begin{figure}[htb]
	\centerfloat{
    \includegraphics[width=1.0\textwidth]{chapter3/ch3_plot_power}
  }
  \caption{Вероятность идентификации метки при сбросе питания каждый раунд (левая колонка) или каждые восемь раундов (правая колонка), опрос по флагу $A$. Идентификация по EPCID или по комбинации EPCID и TID.}
  \label{fig:ch3_plot_power}
\end{figure}

Гораздо выше эффективность системы, если считыватель периодически сбрасывает питание. На рис.~\ref{fig:ch3_plot_power} показаны результаты моделирования для второго и третьего сценариев "--- сброса питания в каждом раунде или каждые восемь раундов. Стоит отметить, результаты при сбросе питания каждые восемь раундов оказались несколько лучше при высоком BER. Это можно объяснить тем, что сброс питания занимает время, поэтому слишком частые сбросы существенно понижают число раундов, в которых успевает принять участие метка.

\begin{figure}[htb]
	\centerfloat{
    \includegraphics[width=1.0\textwidth]{chapter3/ch3_plot_target}
  }
  \caption{Вероятность идентификации метки по EPCID и TID при периодической смене флага опроса.}
  \label{fig:ch3_plot_target}
\end{figure}

Наконец, на рис.~\ref{fig:ch3_plot_target} приведены результаты исследования четвертого и пятого сценариев, в которых считыватель периодически меняет флаги опроса. Результаты показывают, что смена флага позволяет эффективно повысить вероятность идентификации, причем даже лучше, чем приодический сброс питания. Сброс каждые два раунда оказывается эффективней, так как при более редкой смене флага слишком много раундов метки, ошибившиеся в передаче EPCID, пропускают.





%%%%%%%%%%%%%%%%%%%%%%%%%%%%%%%%%%%%%%%%%%%%%%%%%%%%%%%%%%%%%%%%%%%%%%%%%%%%%%%%
\section{Заключение}\label{sec:ch3_conclusion}
%%%%%%%%%%%%%%%%%%%%%%%%%%%%%%%%%%%%%%%%%%%%%%%%%%%%%%%%%%%%%%%%%%%%%%%%%%%%%%%%
В главе были представлены следующие результаты.

\begin{enumerate}
\item Предложена новая аналитическая модель системы радиочастотной идентификации мобильных меток, позволяющая учитывать сбросы питания считывателем и смены флагов сессий. Модель позволяет находить оценки длительностей раундов и вероятности идентификации меток. Модель включает в себя два неоднородных марковских процесса, описывающих число участвующих в каждом раунде меток, и состояние отдельной метки, для которой вычисляется вероятности идентификации. Переходы между состояниями процессов происходят в соотвтетствии с действиями считывателя и изменением числа меток в области чтения.
\item Приведены численные результаты, показывающие, что периодические сбросы питания и смены флагов опросов значительно повышают вероятность идентификации движущейся метки.
\end{enumerate}

\clearpage
