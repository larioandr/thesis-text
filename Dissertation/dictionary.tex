\chapter*{Словарь терминов}             % Заголовок
\addcontentsline{toc}{chapter}{Словарь терминов}  % Добавляем его в оглавление

% \textbf{TeX} : Cистема компьютерной вёрстки, разработанная американским профессором информатики Дональдом Кнутом

% \textbf{панграмма} : Короткий текст, использующий все или почти все буквы алфавита

\textbf{backoff} : алгоритм двоичной экспоненциальной выдержки, процедура выбора случайного времени ожидания начала передачи при возникновении коллизий

\textbf{RFID} : технология радиочастотной идентификации, в которой активное устройство (считыватель) получает идентификатор пассивного устройства (метки) с помощью передачи радиосигналов

$\mathbb{P}\{A\}$ : вероятность события $A$

$\mathbb{M}\xi$ : математическое ожидание случайной величины $\xi$

$\oplus$ : операция Кронекеровой суммы матриц

$\{A\}_{ij}$ : $(i,j)$-й элемент матрицы $A$

$I_W$ : единичная матрица размерности $W$

$\mathbf{0}$ : нулевой вектор-строка

$\mathbf{1}$ : вектор-столбец, состоящий из единиц