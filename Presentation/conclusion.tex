\begin{frame}[allowframebreaks]
    \frametitle{Научная новизна}
    \begin{itemize}
        \item Впервые предложена стохастическая модель системы радиочастотной идентификации с мобильными метками, учитывающая различные законы движения меток, а также сценарии переключения питания и смены опрашиваемых значений флагов сессий.
        \item Предложен комплекс новых аналитических и имитационных моделей для анализа вероятности идентификации быстро движущихся транспортных средств с учетом особенностей логического и физического уровней EPC Gen2, а также учитывающих особенности распространения радиосигналов вблизи автодороги.
        \item Предложена методика построения сетей массового обслуживания с марковскими входными потоками и распределениями фазового типа для адекватного моделирования многошаговых беспроводных сетей, учитывающая различия в трафике и интерференции, испытываемые разными станциями сети.
        \framebreak
        \item Предложен и исследован метод приближенного вычисления оценок характеристик тандемных открытых сетей массового обслуживания с коррелированными марковскими входными потоками и распределениями обслуживания фазового типа, основанный на аппроксимации потоков обслуженных заявок PH-распределениями или MAP-потоками по первым моментам и автокорреляции. В численном эксперименте показано, что предложенный метод позволяет находить оценки характеристик сети с высокой точностью за малое время.
        \item Разработана и реализована распределенная система управления RFID-считывателями, предназначенная для организации сбора данных об идентифицированных транспортных средствах и допускающая объединение с данными, поступающими от прочих источников идентификации.
        \item Проведено три эксперимента по внедрению системы радиочастотной идентификации на дорогах в г. Казань и на ЦКАД в Московской области, получены экспериментальные данные об эффективности разработанной системы.
          \end{itemize}
\end{frame}
\note{
    Проговаривается вслух научная новизна
}

\begin{frame}
    \frametitle{Научная и практическая значимость}
    \begin{itemize}
        \item Получены выражения для \dots.
        \item Определены условия \dots.
        \item Разработаны устройства \dots.
    \end{itemize}
\end{frame}
\note{
    Проговариваются вслух научная и практическая значимость
}

\begin{frame}
    \frametitle{Свидетельство о регистрации программы}
    \begin{figure}[h]
        \centering
        \includegraphics[height=0.7\textheight]{registration}
    \end{figure}
\end{frame}
\note{
    Получено свидетельство о регистрации разработанной программы \textsc{Hello~world™}.
}

\begin{frame}
    \frametitle{Акт о внедрении}
    \begin{figure}[h]
        \centering
        \fbox{
            \begin{minipage}[t]{0.4\linewidth}
                \includegraphics[width=\linewidth]{implementation}
            \end{minipage}
        }
    \end{figure}
\end{frame}
\note{
    Получен акт о внедрении.
}

% \begin{frame} % публикации на одной странице
\begin{frame}[t,allowframebreaks] % публикации на нескольких страницах
    \frametitle{Основные публикации}
    \nocite{RFID_JRFID2017}
    \nocite{WINET_IJPAM2016}
    \nocite{WINET_TCOMM2015}
    \nocite{QS_JPU2013}
    \nocite{QS_JITCS2013}
    \nocite{QS_TCOMM2012}

    %
    %% authorwos
    % \nocite{wosbib1}%

    \nocite{QS_ICAAPSP2020}
    \nocite{QS_ITMM2019}
    \nocite{RFID_IEEERFID2018}
    \nocite{RFID_SYNCHROINFO2018}
    \nocite{RFID_IEEERFID2017}
    \nocite{QS_AICT2017}
    \nocite{QS_ITMM2017}
    \nocite{QS_ITMM2016}
    \nocite{QS_DCCN2016_CCIS}
    \nocite{RFID_DCCN2016_CCIS}
    \nocite{RFIDCTRL_NETS2CARS2014}
    \nocite{RFIDTA2012}

    % \nocite{Fedotov2020}
    % \nocite{Fedotov2020a}
    % \nocite{RFID_VSPU2019}
    % \nocite{WINET_DCCN2018}
    % \nocite{RFIDCTRL_DCCN2017}
    % \nocite{RFID_DCCN2015_RUS}
    % \nocite{QS_DCCN2015}
    % \nocite{QS_ITTMM2015}
    % \nocite{RFIDCTRL_VSPU2014}
    % \nocite{RFID_DCCN2013_RUS}

    %
    %% authorscopus
    % \nocite{scbib1}%
    %
    %% authorconf
    % \nocite{confbib1}%
    % \nocite{confbib2}%
    %
    %% authorother
    % \nocite{bib1}%
    % \nocite{bib2}%
    \ifnumequal{\value{bibliosel}}{0}{
        \insertbiblioauthor
    }{
        \printbibliography%
    }
\end{frame}
\note{
    Результаты работы опубликованы в N печатных изданиях,
    в~т.\:ч. M реферируемых изданиях.
}

\begin{frame}
    \frametitle{Участие в конференциях}
    \begin{itemize}
        \item Научная сессия МГУ, Москва 2013--2015;
        \item \rom{24} Russian Conference (RuC 2014), Obninsk, Russia, 2014
        \item \rom{7} International Conference (IAC 16), Busan, Korea,
              2016;
        \item \rom{28} Other Conference (AC 16), East Lansing, MI USA, 2016;
        \item \dots
    \end{itemize}
\end{frame}
\note{
    Работа была представлена на ряде конференций.
}

\begin{frame}[plain, noframenumbering] % последний слайд без оформления
    \begin{center}
        \Huge
        Спасибо за внимание!
    \end{center}
\end{frame}
