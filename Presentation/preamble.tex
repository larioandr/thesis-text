\begin{frame}[noframenumbering,plain]
    \setcounter{framenumber}{1}
    \maketitle
\end{frame}

\begin{frame}[allowframebreaks]
    \frametitle{Положения, выносимые на защиту}
    \begin{enumerate}
        \item Стохастическая модель системы радиочастотной идентификации ТС, учитывающая скорость движения RFID-меток, расположенных на номерных знаках автомобилей, а также различные сценарии проведения циклического опроса и сбора данных с меток.
        \item Новый комплекс аналитических и имитационных моделей для анализа вероятности идентификации ТС, учитывающих особенности логического и физического уровней протокола стандарта EPC Class 1 Gen.2, и особенности распространения радиосигналов между RFID-меткой и считывателем.
        \item Новая методика моделирования многошаговых беспроводных сетей с помощью тандемных сетей массового обслуживания, учитывающая особенности трафика и интерференции в каналах связи.
        \framebreak
        \item Метод вычисления оценок характеристик многофазных систем массового обслуживания большой размерности с коррелированными входными потоками и распределениями обслуживания фазового типа.
        \item Архитектура и реализация новой распределенной системы управления RFID-считывателями, предназначенная для организации сбора данных об идентифицированных транспортных средствах.
        \item Результаты экспериментальных внедрений разработанной системы радиочастотной идентификации транспортных средств на автодорогах в г. Казань и в Московской области.
    \end{enumerate}
\end{frame}
\note{
    Проговариваются вслух положения, выносимые на защиту
}

\begin{frame}
    \frametitle{Содержание}
    \tableofcontents
\end{frame}
\note{
    Работа состоит из пяти глав.

    \medskip
    В первой главе \dots

    Во второй главе \dots

    Третья глава посвящена \dots

    В четвёртой главе \dots
}
