\begin{frame}[noframenumbering,plain]
    \setcounter{framenumber}{1}
    \maketitle
\end{frame}

\begin{frame}
    \frametitle{Актуальность}
    \footnotesize
    \begin{itemize}
        \item Дорожно-транспортные происшествия являются причиной большого количества смертей, ранений и наносят значительный экономический ущерб как в России, так и во всем мире. Так, в 2020 году в России в ДТП погибло свыше 16 тыс. человек и еще 183 тыс. получили ранения. Одна из основных причин -- нарушение правил дорожного движения.
        \item Системы видеофиксации нарушений теряют эффективность при плохих погодных условиях или при загрязненных или скрытых номерных знаках. Возможное решение -- использование технологии RFID для идентификации транспорта, исследуемое в диссертации.
        \item Для своевременной реакции на нарушения, поиска угнанных транспортных средств и решения прочих задач необходимо соединение точек идентификации с центрами обработки данных быстрыми сетями. Это обуславливает актуальность исследования производительности беспроводных сетей, которые могут эффективно использоваться для подключения точек идентификации к центрам обработки данных.
        \item Для эффективного управления системой идентификации, включающей множество точек идентификации, нужны современные распределенные компьютерные системы управления, что делает актуальным исследование и разработку таких систем.
        \item Актуальность диссертационного исследования также подтверждается постановлением Правительства РФ о создании пилотных зон систем радиочастотной идентификации транспорта в Москве, Санкт-Петербурге и Казани в 2022--2024 годах.
    \end{itemize}
\end{frame}
\note{
    Проговаривается вслух научная новизна
}

\begin{frame}
    \frametitle{Постановка задач исследования}
    \begin{enumerate}
        \item Разработка и исследование комплекса аналитических и имитационных моделей для анализа и оптимизации основных характеристик систем радиочастотной идентификации транспортных средств.
        \item Разработка методики оценки производительности широкополосных беспроводных сетей, использующихся для передачи данных от RFID-считывателей в центры обработки данных, на основе методов теории массового обслуживания и марковских случайных процессов.
        \item Разработка архитектуры и реализация распределенной компьютерной системы управления и сбора данных с RFID-считывателей, ее экспериментальное внедрение и проведение испытаний.
    \end{enumerate}
\end{frame}


\begin{frame}[allowframebreaks]
    \frametitle{Научная новизна}
    \small
    \begin{itemize}
        \item Впервые предложена и исследована стохастическая модель системы радиочастотной идентификации ТС, учитывающая скорость движения RFID-меток, расположенных на номерных знаках автомобилей, а также различные сценарии проведения опроса и сбора данных с меток.
        \item Разработан комплекс новых аналитических и имитационных моделей для анализа вероятности идентификации ТС, учитывающих особенности логического и физического уровней протокола стандарта EPC Class 1 Gen.2, и особенности распространения радиосигналов между RFID-меткой и считывателем.
        \item Предложена новая методика моделирования многошаговых беспроводных сетей с помощью тандемных сетей массового обслуживания, учитывающая особенности трафика и интерференции в каналах связи.
        \item Разработан оригинальный метод вычисления оценок характеристик многофазных систем массового обслуживания большой размерности с коррелированными входными потоками и распределениями обслуживания фазового типа.
        \break
        \item Разработана архитектура и реализована новая распределенная система управления RFID-считывателями, предназначенная для организации сбора данных об идентифицированных транспортных средствах.
        \item Проведена обработка экспериментальных данных, полученных при опытных внедрениях разработанной системы в г. Казань и на ЦКАД в Московской области, показавшая высокое совпадение с теоретическими результатами диссертации.
        \item Новизна также подтверждается приведенным в диссертации обзором научных исследований, включающим 190 источников.
    \end{itemize}
\end{frame}
\note{
    Проговаривается вслух научная новизна
}

\begin{frame}[allowframebreaks]
    \frametitle{Практическая значимость}
    \small
    \begin{itemize}
        \item Аналитические и имитационные модели и методы, предложенные в диссертации, могут эффективно использоваться для оценки производительности систем радиочастотной идентификации автомобилей и широкополосных беспроводных сетей.
        \item Распределенная система управления считывателями и программное обеспечение, описанные в работе, использовались в трех экспериментальных внедрениях на автодорогах в г. Казань и Московской области.
        \item Практическая значимость диссертационной работы подтверждается актами о внедрении, полученными от Государственного бюджетного учреждения <<Безопасность дорожного движения>> (г. Казань) и ПАО <<Микрон>>.
    \end{itemize}
    \framebreak
    Результаты работы также были использованы в исследованиях, проводимых по следующим грантам:
    \footnotesize
    \begin{itemize}
        \item Контракт c Министерством образования и науки РФ № 14.514.11.4071 в рамках федеральной целевой программы <<Исследования и разработки по приоритетным направлениям развития научно-технологического комплекса России на 2007-2013 годы>>.
        \item Соглашение с Министерством образования и науки РФ о предоставлении субсидии от 22.10.2014 г. № 14.613.21.0020 в рамках федеральной целевой программы <<Исследования и разработки по приоритетным направлениям развития научно-технологического комплекса на 2014-2020 годы>>.
        \item Грант Российского научного фонда (РНФ) № 16-49-02021.
        \item Грант РФФИ № 13-07-00737.
        \item Гранты РФФИ (международный проект РФФИ "--- БРФФИ) № 14-07-90015, 16-57-00130
    \end{itemize}
\end{frame}
\note{
    Проговариваются вслух научная и практическая значимость
}


\begin{frame}[allowframebreaks]
    \frametitle{Положения, выносимые на защиту}
    \begin{enumerate}
        \item Стохастическая модель системы радиочастотной идентификации ТС, учитывающая скорость движения RFID-меток, расположенных на номерных знаках автомобилей, а также различные сценарии проведения циклического опроса и сбора данных с меток.
        \item Новый комплекс аналитических и имитационных моделей для анализа вероятности идентификации ТС, учитывающих особенности логического и физического уровней протокола стандарта EPC Class 1 Gen.2, и особенности распространения радиосигналов между RFID-меткой и считывателем.
        \item Новая методика моделирования многошаговых беспроводных сетей с помощью тандемных сетей массового обслуживания, учитывающая особенности трафика и интерференции в каналах связи.
        \framebreak
        \item Метод вычисления оценок характеристик многофазных систем массового обслуживания большой размерности с коррелированными входными потоками и распределениями обслуживания фазового типа.
        \item Архитектура и реализация новой распределенной системы управления RFID-считывателями, предназначенная для организации сбора данных об идентифицированных транспортных средствах.
        \item Результаты экспериментальных внедрений разработанной системы радиочастотной идентификации транспортных средств на автодорогах в г. Казань и в Московской области.
    \end{enumerate}
\end{frame}
\note{
    Проговариваются вслух положения, выносимые на защиту
}

\begin{frame}
    \frametitle{Структура диссертации и публикации}
    \small
    Работа состоит из введения, заключения, одного приложения и пяти глав:
    \begin{enumerate}
        \item Распределенная система радиочастотной идентификации.
        \item Исследование производительности систем радиочастотной идентификации.
        \item Аналитическая модель системы радиочастотной идентификации.
        \item Анализ производительности опорной беспроводной сети.
        \item Разработка и экспериментальное внедрение системы радиочастотной идентификации.
    \end{enumerate}

    Полный объем диссертации составляет 200 страниц, включая 85 рисунков и 8 таблиц. Список литературы содержит 210 наименований.
    \vfill
    Основные результаты по теме диссертации изложены в 20 печатных изданиях, 4 из которых изданы в журналах, рекомендованных ВАК, 15—в изданиях, индексируемых Web of Science и Scopus, 1—в издании, индексируемом РИНЦ.

\end{frame}
\note{
    Работа состоит из пяти глав.

    \medskip
    В первой главе \dots

    Во второй главе \dots

    Третья глава посвящена \dots

    В четвёртой главе \dots
}
