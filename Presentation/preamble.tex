\begin{frame}[noframenumbering,plain]
    \setcounter{framenumber}{1}
    \maketitle
\end{frame}

\begin{frame}
    \frametitle{Положения, выносимые на защиту}
    \begin{itemize}
        \item Стохастическая модель системы радиочастотной идентификации с мобильными метками, учитывающая сценарии переключения питания и смены опрашиваемых значений флагов сессий.
        \item Комплекс аналитических и имитационных моделей для анализа вероятности идентификации быстро движущихся транспортных средств с учетом особенностей логического и физического уровней EPC Gen2.
        \item Метод построения моделей открытых тандемных сетей массового обслуживания для анализа производительности многошаговых беспроводных сетей.
        \item Метод расчета оценок характеристик открытых сетей массового обслуживания с марковскими потоками и обслуживанием фазового типа с использованием аппроксимации потоков обслуженных пакетов PH-распределениями и MAP-потоками.
        \item Реализация распределенной системы управления RFID-считывателями, используемой для создания систем радиочастотной идентификации автотранспорта.    \end{itemize}
\end{frame}
\note{
    Проговариваются вслух положения, выносимые на защиту
}

\begin{frame}
    \frametitle{Содержание}
    \tableofcontents
\end{frame}
\note{
    Работа состоит из пяти глав.

    \medskip
    В первой главе \dots

    Во второй главе \dots

    Третья глава посвящена \dots

    В четвёртой главе \dots
}
